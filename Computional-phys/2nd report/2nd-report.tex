\documentclass[12pt, a4paper, oneside]{ctexart}
\usepackage{amsmath, amsthm, amssymb, graphicx}
\usepackage[bookmarks=true, colorlinks, citecolor=blue, linkcolor=black]{hyperref}
\usepackage[margin = 25mm]{geometry}
\usepackage{setspace}
\usepackage{listings}
\usepackage{ctex}
\usepackage{float}
\usepackage{xcolor}

\definecolor{codegreen}{rgb}{0,0.6,0}
\definecolor{codegray}{rgb}{0.5,0.5,0.5}
\definecolor{codepurple}{rgb}{0.58,0,0.82}
\definecolor{backcolour}{rgb}{0.95,0.95,0.92}

\lstdefinestyle{mystyle}{
    backgroundcolor=\color{backcolour},   
    commentstyle=\color{codegreen},
    keywordstyle=\color{magenta},
    numberstyle=\tiny\color{codegray},
    stringstyle=\color{codepurple},
    basicstyle=\ttfamily\footnotesize,
    breakatwhitespace=false,         
    breaklines=true,                 
    captionpos=b,                    
    keepspaces=true,                 
    numbers=left,                    
    numbersep=5pt,                  
    showspaces=false,                
    showstringspaces=false,
    showtabs=false,                  
    tabsize=2
}

\lstset{style=mystyle}


\title{使用隐式欧拉法求解常微分方程问题}
\date{\today}
\author{202011010101物理2001孙陶庵}
\begin{document}
\begin{spacing}{2.0}
\tableofcontents
\maketitle

\section{问题}
对于下列微分方程初值问题:
\begin{center}
$    \left\{\begin{matrix} 
        \frac{\mathrm{d}y}{\mathrm{d}x}=xy^2+2y \\  
        y(0)=-5 
      \end{matrix}\right. $
\end{center}
使用隐式欧拉法求解其在$0<x<5$时的数值解。
\section{求解方程:解析解}
透过python的sympy库求解
\begin{lstlisting}[language=Python, caption=0.1s]
    from sympy import Function, dsolve, Eq, Derivative, sin, cos, symbols
    from sympy.abc import x
    f = Function('f')
    dsolve(Derivative(f(x), x) - x * f(x)**2 -2 * f(x), f(x), ics={f(0): -5})
    
\end{lstlisting}
可以得到方程解为:
\begin{center}
    $f{\left(x \right)} = \frac{4 e^{2 x}}{- 2 x e^{2 x} + e^{2 x} - \frac{9}{5}}$
\end{center}
\section{透过三种不同方法求解微分方程}
\subsection{隐式欧拉法}
\begin{lstlisting}[language=MATLAB, caption=隐式欧拉法]
    x0 = 0;
    y0 = -5;
    a = 0;
    b = 5;
    h = 0.2;
    y = Euler_Implicit(@f, y0, a, b, h);
    
    x = linspace(a, b, length(y));
    hold on
    plot(x, y, '-o');
    plot(x,RealFunc(x),"-b");
    xlabel('x');
    ylabel('y');
    title('Numerical Solution and error');

    function y = Euler_Implicit(f, y0, a, b, h)
        n = round((b-a)/h);
        y = zeros(1, n+1);
        y(1) = y0;
        
        for i = 2:n+1
            xi = a + (i-1) * h;
            yi = y(i-1);
            for j = 1:100
                yi = y(i-1) + h * f(xi, yi);
            end
            y(i) = yi;
        end
    end
    
    function f = f(x, y)
         f= x.*y.^2 + 2.*y;
    end
    
    function ye = RealFunc(x)
        ye = (20.*exp(2.*x))./(exp(2.*x)-10.*x.*exp(2.*x)-9);
    end
    
\end{lstlisting}
其中$function ye = RealFunc(x)$部分是文章第一部分所解出的解析解;$function f = f(x, y)$部分是原方程,将$\frac{\mathrm{d}y}{\mathrm{d}x}$替换为f;
$function y = Euler\_Implicit(f, y0, a, b, h)$是函数主体,解释如下:\\
1. 根据步长 h 计算需要多少个点n,并初始化结果数组;\\
2. 将初始值赋给第一个元素;\\
3. 循环从 i=2 到 i=n+1,每次计算 xi 和 yi;\\
4. 在内部循环中进行100次迭代,利用欧拉隐式法更新 yi 的数值;\\
5. 将最终得到的 yi 赋给结果数组。\\
\begin{figure}[htbp][H]
    \centering
    \includegraphics[width=8cm]{Eular-IM.jpg}
    \caption{隐式欧拉法}
\end{figure}

\subsection{显式欧拉法}
\begin{lstlisting}[language=MATLAB, caption=显式欧拉法]
    x0 = 0;
    y0 = -5;
    h = 0.1;
    y = Euler_Explicit(@f, y0, h);
    
    x = linspace(0, 5, length(y));
    hold on
    plot(x, y, '-o');
    plot(x,RealFunc(x),"-b");
    
    xlabel('x');
    ylabel('y');
    title('Numerical Solution and Error');
    legend('Numerical Solution');
    
    
    function y = Euler_Explicit(f, y0, h)
        t0 = 0;
        tp = 5;
        t = t0:h:tp;
        y = zeros(size(t));
        y(1) = y0;
    
        for i = 1:length(t)-1
            y(i+1) = y(i) + h * f((i+1) * h, y(i));
        end
    end
    
    function fullderi = f(x, y)
        fullderi = x.*y.^2 + 2*y;
    end
    
    function ye = RealFunc(x)
        ye = (20.*exp(2.*x))./(exp(2.*x)-10.*x.*exp(2.*x)-9);
    end    
\end{lstlisting}
$    function y = Euler\_Explicit(f, y0, h)$是函数主体,解释如下:
1.初始化变量$x_0$、$y_0$和$h$ \\
2.调用$Euler_Explicit$函数计算数值解$y$ \\
3.生成等间隔的自变量$x$ \\
4.绘制数值解曲线和真实解曲线,并添加标签、标题及图例。 \\
其中,$Euler_Explicit$函数使用欧拉显式法对微分方程进行离散化处理并求出数值近似解;$f(x,y)$为给定的微分方程右侧;$RealFunc(x)$为该微分方程的真实解。
\begin{figure}[htbp][H]
    \centering
    \includegraphics[width=8cm]{EE.jpg}
    \caption{显式欧拉法}
\end{figure}

\subsection{4th-order Runge-Kutta}
\begin{lstlisting}[language=MATLAB, caption=4th-order Runge-Kutta]
    y0 = -5;
    t0 = 0;
    tn = 5;
    h = 0.2;
    
    result = Runge_Kutta41(@f, t0, y0, tn, h);
    x = linspace(0, 5, length(result));
    hold on
    plot(x, result, '-o');
    plot(x,RealFunc(x),"-b");
    
    xlabel('x');
    legend();
    
    function res = Runge_Kutta41(f, t0, y0, tn, h)
        res = [y0];
        t = t0;
        for i = 1:floor((tn-t0)/h)
            k1 = f(t, res(end));
            k2 = f(t + h / 2, res(end) + h * k1 / 2);
            k3 = f(t + h / 2, res(end) + h * k2 / 2);
            k4 = f(t + h, res(end) + h * k3);
            y_next = res(end) + h*(k1 + 2 * k2 + 2 * k3 + k4)/6;
            res = [res, y_next];
            t = t + h;
        end
    end
    
    function y = f(x, y)
        y = x*y^2 + 2*y;
    end
    
    function ye = RealFunc(x)
        ye = (20.*exp(2.*x))./(exp(2.*x)-10.*x.*exp(2.*x)-9);
    end
    
\end{lstlisting}
\begin{figure}[htbp][H]
    \centering
    \includegraphics[width=8cm]{Runge_Kutta411.jpg}
    \caption{4th-order Runge-Kutta}
\end{figure}

这部分比较简单,主要是透过以下公式进行循环而得:
\begin{center}
    
    $    k_1 = \ f(t_n, y_n)$, \\
    $k_2 = \ f\!\left(t_n + \frac{h}{2}, y_n + h\frac{k_1}{2}\right)$, \\ 
    $k_3 = \ f\!\left(t_n + \frac{h}{2}, y_n + h\frac{k_2}{2}\right)$, \\
    $k_4 = \ f\!\left(t_n + h, y_n + hk_3\right)$.\\
    $y_{n+1} = y_n + \frac{1}{6}\left(k_1 + 2k_2 + 2k_3 + k_4 \right)$h,\\
    $t_{n+1} = t_n + h$ \\


\end{center}
\section{处理三种方法的误差-代码实现}
主要是分析误差。
\subsection{隐式欧拉法}
\begin{lstlisting}[language=Python, caption=隐式欧拉法]
    import numpy as np
    import matplotlib.pyplot as plt
    
    
    def Euler_Implicit(f, y0, a, b, h):
        n = round((b-a)/h)
        y = np.zeros(n+1)
        y[0] = y0
    
        for i in range(1, n+1):
            xi = a + i * h
            yi = y[i - 1]
            for j in range(10):
                yi = y[i - 1] + h * f(xi, yi)
            y[i] = yi
        
        return y
    
    
    def error(f, y0, a, b, h):
    
        def y_exact(x):
            return 20*np.exp(2*x)/(np.exp(2*x) - 10*x*np.exp(2*x) - 9)
    
        y_num = Euler_Implicit(f, y0, a, b, h)
        x = np.arange(a, b + h, h)[:len(y_num)]
        e = y_num - y_exact(x)
    
        return x, e
    
    
    def f(x, y):
        return x*y**2+2*y
    
    
    x0 = 0
    y0 = -5
    a = 0
    b = 5
    h = 0.1
    y = Euler_Implicit(f, y0, a, b, h)
    
    error_list = []
    ha = np.arange(0.1, 0.7, 0.01)
    for h in np.arange(0.1, 0.7, 0.01):
        
        y = Euler_Implicit(f, y0, a, b, h)
        x, e = error(f, y0, a, b, h)
        x = np.linspace(0, 5, len(y))
        error_list.append(e[-1])
        print(h,e[-1])
    
    
    plt.plot(ha, error_list)
    plt.xlabel('h')
    plt.ylabel('e[-1]')
    plt.title('Error vs. Step Size')
    plt.show() 
    
\end{lstlisting}
\subsection{显式欧拉法}
\begin{lstlisting}[language=Python, caption=显式欧拉法]
    import numpy as np
    import matplotlib.pyplot as plt
    
    def Euler_Explicit(f, y0, h):
        t0 = 0
        tp = 5
        t = np.arange(t0, tp+h, h)
        y = np.zeros(len(t))
        y[0] = y0
    
        for i in range(0, len(t)-1):
            y[i+1] = y[i] + h * f((i+1) * h, y[i])
        return y
    
    def error(f, y0, h):
        t0 = 0
        tp1 = 5
        
        def y_exact(x):
            return 20*np.exp(2*x)/(np.exp(2*x) - 10*x*np.exp(2*x) - 9)
    
        y_num = Euler_Explicit(f, y0, h)
        x = np.arange(t0, tp1 + h, h)
        e = y_num - y_exact(x)
    
        return x, e
        
    def f(x, y):
        return x*y**2+2*y
    x0 = 0
    y0 = -5
    
    error_list = []
    ha = np.arange(0.26, 1.5, 0.01)
    for h in np.arange(0.26, 1.5, 0.01):
        
        y = Euler_Explicit(f, y0, h)
        x, e = error(f, y0, h)
        x = np.linspace(0, 5, len(y))
        error_list.append(e[-1])
        print(h,e[-1])
    
    xdata = ha
    ydata = np.array(error_list)
    coef = np.polyfit(xdata, ydata, 2)
    f_fit = np.poly1d(coef)
    xfit = np.linspace(xdata[0], xdata[-1], 100)
    yfit = f_fit(xfit)
        
    plt.plot(ha, error_list, 'bo', label='data')
    plt.plot(xfit, yfit, 'r-', label='fit')
    plt.xlabel('h')
    plt.ylabel('e[-1]')
    plt.title('Error vs. Step Size')
    plt.legend()
    plt.show()
    plt.plot(ha, error_list)
    plt.xlabel('h')
    plt.ylabel('e[-1]')
    plt.title('Error vs. Step Size')
    plt.show()
    
\end{lstlisting}
\subsection{4th-order Runge-Kutta}
\begin{lstlisting}[language=Python, caption=4th-order Runge-Kutta]
    import numpy as np
    import matplotlib.pyplot as plt
    
    def Runge_Kutta4(f, t0, y0, tn, h):
        res = [y0]
        t = t0
        for i in range(int((tn-t0)/h)):
            k1 = f(t, res[-1])
            k2 = f(t + h / 2, res[-1] + h * k1 / 2)
            k3 = f(t + h / 2, res[-1] + h * k2 / 2)
            k4 = f(t + h, res[-1] + h * k3)
            y_next = res[-1] + h * (k1 + 2 * k2 + 2 * k3 + k4)/6
            res.append(y_next)
            t += h
        return res
    
    def f(x, y):
        return x*y**2+2*y
    
    def error(f, y0, t0, tn, h):
    
        def y_exact(t):
            return 20*np.exp(2*x)/(np.exp(2*x) - 10*x*np.exp(2*x) - 9)
    
        y_num = Runge_Kutta4(f, t0, y0, tn, h)
        t = np.arange(t0, tn + h, h)
        e = y_num - y_exact(t)
    
        return t, e
    
    y0 = -5
    t0 = 0
    tn = 5
    h = 0.1
    
    result = Runge_Kutta4(f, t0, y0, tn, h)
    x = np.linspace(0, 5, len(result))
    plt.plot(x, result, '-o',label='Numerical Solution')
    
    t, e = error(f, y0, t0, tn, h)
    
    plt.plot(t, e, '-o', label='Error')
    
    plt.xlabel('t')
    plt.title('Numerical Solution and Error')
    plt.legend()
    plt.show()
    
\end{lstlisting}
\section{处理三种方法的误差-比较}
\begin{figure}[H]
    \begin{minipage}[t]{0.5\linewidth}
        \centering
        \includegraphics[scale=0.3]{EIM-ddd.png}
        \caption{隐式欧拉法的误差}
        \label{fig:side:a}
      \end{minipage}%
      \begin{minipage}[t]{0.5\linewidth}
        \centering
        \includegraphics[scale=0.3]{EE-ddd.png}
        \caption{显式欧拉法的误差}
        \label{fig:side:b}
      \end{minipage}
      \begin{minipage}[t]{0.5\linewidth}
        \centering
        \includegraphics[scale=0.3]{Runge_Kutta4 ddd.png}
        \caption{4th-order Runge-Kutta的误差}
        \label{fig:side:b}
      \end{minipage}
\end{figure}
\subsection{误差分析}
我将数据处理的档案加到附件里
输出结果为:
\begin{center}
    EE.txt: 最接近 0 的值是 [0.26, -0.008873641307191593],其距离为 0.2688736413071916。\\
    EIM.txt: 最接近 0 的值是 [0.1, -0.03759539961851743],其距离为 0.13759539961851744。\\
    runge.txt: 最接近 0 的值是 [0.44999999999999984, -0.06431906956574363],其距离为 0.5143190695657435。
\end{center}
比较上面三张图片及处理完的数据可以看出隐式欧拉法(EIM)在步长的选取上可以选择的精度最小,说明在这三种方法中最为精确。
同时,我们也可以通过这个图形来选择一个合适的步长,使得数值解的绝对误差达到一个满意的精度水平。
\subsection{收敛性分析}
\begin{figure}[H]
    \begin{minipage}[t]{0.5\linewidth}
        \centering
        \includegraphics[scale=0.3]{EE-dfit.png}
        \caption{显式欧拉法}
        \label{fig:side:a}
      \end{minipage}%
      \begin{minipage}[t]{0.5\linewidth}
        \centering
        \includegraphics[scale=0.3]{Runge_Kutta4-fit.png}
        \caption{Runge}
        \label{fig:side:b}
      \end{minipage}
\end{figure}

收敛应该指的是随着步长 $h$ 的减小,数值解 $y_n$ 逐渐接近精确解 $y(x_n)$,即 $\lim_{h \to 0} = y(x_n) $。

为了判断一个数值方法是否收敛,可以考虑不同步长 $h$ 下的数值解和精确解之间的误差,并观察误差随着步长 $h$ 的变化情况。

如果误差随着 $h$ 的减小而减小,那么这个数值方法就是收敛的。

图形反映了步长(h)和数值解的误差之间的关系。横轴表示步长(h),纵轴表示数值解的绝对误差。
通过这个图形,我们可以了解到,当步长减小时,数值解的绝对误差会逐渐减小。

拟合直线的斜率可以表示误差随步长$h$的变化率,因此可以通过拟合直线来估计误差的收敛阶。
其绝对值越小,误差随步长$h$的变化越慢,收敛阶就越高。拟合直线截距的意义则是误差的常数项。
如果误差本身存在常数项,那么通过拟合直线的截距就可以得到误差的常数项大小。
也就是误差与步长的关系,从而评估数值方法的收敛速度和精度。通过拟合出的直线,我们可以得到收敛阶的估计,
也就是该数值方法在每次步长减小一半时误差的减小速率。一般来说,收敛阶越高,数值方法的收敛速度越快,精度越高。




\end{spacing}{}

\end{document}