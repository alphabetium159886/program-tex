\documentclass[12pt, a4paper, oneside]{ctexart}
\usepackage{amsmath, amsthm, amssymb, graphicx}
\usepackage[bookmarks=true, colorlinks, citecolor=blue, linkcolor=black]{hyperref}
\usepackage[margin = 25mm]{geometry}
\usepackage{setspace}
\usepackage{listings}
\usepackage{ctex}
\usepackage{float}
\usepackage{xcolor}
\usepackage{amsmath}
\definecolor{codegreen}{rgb}{0,0.6,0}
\definecolor{codegray}{rgb}{0.5,0.5,0.5}
\definecolor{codepurple}{rgb}{0.58,0,0.82}
\definecolor{backcolour}{rgb}{0.95,0.95,0.92}

\lstdefinestyle{mystyle}{
    backgroundcolor=\color{backcolour},   
    commentstyle=\color{codegreen},
    keywordstyle=\color{magenta},
    numberstyle=\tiny\color{codegray},
    stringstyle=\color{codepurple},
    basicstyle=\ttfamily\footnotesize,
    breakatwhitespace=false,         
    breaklines=true,                 
    captionpos=b,                    
    keepspaces=true,                 
    numbers=left,                    
    numbersep=5pt,                  
    showspaces=false,                
    showstringspaces=false,
    showtabs=false,                  
    tabsize=2
}

\lstset{style=mystyle}


\title{一维有限元方法求解常微分方程边值问题}
\date{\today}
\author{202011010101物理2001孙陶庵}
\begin{document}
\begin{spacing}{2.0}
\tableofcontents
\maketitle

\section{问题}
使用一维有限元方法求解下列常微分方程边值问题:
$\displaystyle \left\{\begin{matrix}y''-y+x=0,\quad x\in\left[1,2\right]\\ y\left(1\right)=1,y\left(2\right)=3\end{matrix}\right.$

\section{方法推导}
1.离散化区间[1,2],选择有限元网格。可以选择等距节点网格,即将[1,2]等分为n个子区间,每个子区间内选择一个节点。这里选择n=4,即将[1,2]等分为4个子区间,
每个子区间内选择一个节点,得到节点序列[1, 1.333, 1.667, 2]。
\\
2.根据所选有限元网格,建立有限元函数空间。由于是一维问题,可以采用线性元,即每个子区间内用一次多项式近似解。
定义有限元函数空间为$\displaystyle V_h={v_h|v_h(x)=\sum_{j=1}^{n}c_j\varphi_j(x),c_j\in R}$,其中$\varphi_j(x)$为基函数,可以选择线性插值函数。
则有
$\varphi_1(x)=\begin{cases}
    1-\frac{x-x_2}{x_1-x_2}, & x\in[x_1,x_2]\\ % 注意每个分段函数的结尾都应该有一个 \
    0, & \text{otherwise}
\end{cases}$

$\varphi_2(x)=\begin{cases}
    \frac{x-x_1}{x_2-x_1}, & x\in[x_1,x_2]\\
    \frac{x_3-x}{x_3-x_2}, & x\in[x_2,x_3]\\
    0, & \text{otherwise}
\end{cases}$

$\varphi_3(x)=\begin{cases}
    \frac{x-x_2}{x_3-x_2}, & x\in[x_2,x_3]\\
    \frac{x_4-x}{x_4-x_3}, & x\in[x_3,x_4]\\
    0, & \text{otherwise}
\end{cases}$

$\varphi_4(x)=\begin{cases}
    \frac{x-x_3}{x_4-x_3}, & x\in[x_3,x_4]\\
    0, & \text{otherwise}
\end{cases}$
\\
3.将原方程转化为弱形式。对于任意$v_h\in V_h$,将原方程两边乘$v_h$,并在区间[1,2]上积分,得到$\int_1^2(y''-y+x)v_hdx=0$。由于$v_h$是连续线性函数,
可以将积分区间[1,2]上的积分转化为每个子区间内的积分,即$\displaystyle\sum_{k=1}^{3}\int_{x_k}^{x_{k+1}}(y''-y+x)v_hdx=0$。
\\
4.对于每个子区间[k,k+1],将$y(x)$和$v_h(x)$在该区间内分别用基函数展开,即$\displaystyle y(x)=\sum_{j=1}^{2}y_j\varphi_j(x)$,$ \displaystyle v_h(x)=\sum_{j=1}^{2}v_j\varphi_j(x)$。
将$y(x)$和$v_h(x)$在每个子区间[k,k+1]内分别用基函数展开,即用基函数$\varphi_j(x)$的线性组合来近似表示$y(x)$和$v_h(x)$。其中,$y_j$和$v_j$是在
子区间[k,k+1]内的系数,需要确定。因此,在每个子区间上,有$\displaystyle y(x)=\sum_{j=1}^{2}y_j\varphi_j(x)$和$\displaystyle v_h(x)=\sum_{j=1}^{2}v_j\varphi_j(x)$的展开形式。
这里选择的是线性元,所以每个子区间内用一次多项式近似解,即选择两个节点作为基函数的控制点,也就是基函数在这两个点处取值为1和0,其余点处线性插值。

\section{问题}



\end{spacing}{}

\end{document}