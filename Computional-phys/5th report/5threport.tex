\documentclass[12pt, a4paper, oneside]{ctexart}
\usepackage{amsmath, amsthm, amssymb, graphicx}
\usepackage[bookmarks=true, colorlinks, citecolor=blue, linkcolor=black]{hyperref}
\usepackage[margin = 25mm]{geometry}
\usepackage{setspace}
\usepackage{listings}
\usepackage{ctex}
\usepackage{float}
\usepackage{xcolor}
\usepackage{amsmath}
\definecolor{codegreen}{rgb}{0,0.6,0}
\definecolor{codegray}{rgb}{0.5,0.5,0.5}
\definecolor{codepurple}{rgb}{0.58,0,0.82}
\definecolor{backcolour}{rgb}{0.95,0.95,0.92}

\lstdefinestyle{mystyle}{
    backgroundcolor=\color{backcolour},   
    commentstyle=\color{codegreen},
    keywordstyle=\color{magenta},
    numberstyle=\tiny\color{codegray},
    stringstyle=\color{codepurple},
    basicstyle=\ttfamily\footnotesize,
    breakatwhitespace=false,         
    breaklines=true,                 
    captionpos=b,                    
    keepspaces=true,                 
    numbers=left,                    
    numbersep=5pt,                  
    showspaces=false,                
    showstringspaces=false,
    showtabs=false,                  
    tabsize=2
}

\lstset{style=mystyle}


\title{蒙特卡罗伊辛模型}
\date{\today}
\author{202011010101物理2001孙陶庵}
\begin{document}
\begin{spacing}{2.0}
\tableofcontents
\maketitle

\section{问题}
1.模型定义:定义一个2维的N*N伊辛模型,每一个点上有一个自旋,可上可下,体系的哈密顿量为
$\displaystyle H=-J\sum_{<i,j>}s_{i}s_{j}$
其中的$s_{i}$为第i个格点上的自选为1或-1,j为交换常数,模拟中设为1,<i,j>表示对所有最近邻格点求和。
2.Metropolis算法:使用Metropolis算法来模拟这个伊辛模型。在每一步中,随机选择一个格点,计算反转这个格点的自选后能量变化为$\Delta E$,如果$\Delta E<0$则接受这个反转;如果$\Delta E>0$则以概率$exp(\frac{\Delta E}{kT})$接受这个反转,其中k为boltzmann constant,T为温度
3.在不同温度下运行模拟程序,计算并绘制体系的序参量以及其涨落随着温度的变化
\section{}

\begin{lstlisting}[language=Matlab, caption={代码}]

    N = 1000000;
    n = 100;
    J = 1;
    temperatures = linspace(1, 10, 10);  % 不同的温度
    
    monte_carlo_ising_model(N, n, J, temperatures);
    
    
    function z = f(k, T, dE)
        z = exp(-dE / (k * T));
    end
    
    function monte_carlo_ising_model(N, n, J, temperatures)
        SPINARRAY = zeros(length(temperatures), 1);
        order_params = zeros(N, length(temperatures));
    
        for k = 1:length(temperatures)
            T = temperatures(k);
            array = zeros(N, 1);
            S = 2 * randi([0, 1], n, n) - 1;
    
            for i = 1:N
                index = randi(n, 1, 2);
                x1 = mod(index(1) - 2, n) + 1;
                x2 = index(1);
                x3 = mod(index(1), n) + 1;
                y1 = mod(index(2) - 2, n) + 1;
                y2 = index(2);
                y3 = mod(index(2), n) + 1;
                dE = 2 * J * S(x2, y2) * (S(x1, y2) + S(x3, y2) + S(x2, y1) + S(x2, y3));
    
                if dE > 0
                    a = rand();
                    if a < f(1, T, dE)
                        S(x2, y2) = -1;
                    end
                else
                    S(x2, y2) = -S(x2, y2);
                end
    
                ss = sum(sum(S)) / (n * n);
                array(i) = ss;
            end
    
            array1 = array .^ 2;
            MEANSQ1 = mean(array1(floor(N * 3 / 10):end));  % 平方和平均值
            MEANSQ = mean(array(floor(N * 3 / 10):end)) .^ 2;  % 平均平方
            ANS = MEANSQ1 - MEANSQ;  % 序参量的涨落
            SPINARRAY(k) = ANS;
            order_params(:, k) = array;
        end
    
        figure;
        subplot(2, 1, 1);
        plot(1:N, order_params);
        xlabel('MC Steps');
        ylabel('Order Parameter');
        title('Order Parameter vs MC Steps');
        legend(string(temperatures), 'Location', 'northwest');
    
        subplot(2, 1, 2);
        plot(temperatures, SPINARRAY, 'ro-');
        xlabel('Temperature');
        ylabel('Fluctuation');
        title('Fluctuation of Order Parameter vs Temperature');
    end
    
    
\end{lstlisting}



\end{spacing}{}

\end{document}