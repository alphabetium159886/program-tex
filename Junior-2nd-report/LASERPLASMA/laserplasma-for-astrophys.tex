\documentclass[12pt, a4paper, oneside]{ctexart}
\usepackage{amsmath, amsthm, amssymb, graphicx}
\usepackage[bookmarks=true, colorlinks, citecolor=blue, linkcolor=black]{hyperref}
\usepackage[margin = 25mm]{geometry}
\usepackage{setspace}
\usepackage{listings}
\usepackage{ctex}
\usepackage{cite}
\usepackage{float}
\usepackage{xcolor}
\usepackage{amsmath}
\definecolor{codegreen}{rgb}{0,0.6,0}
\definecolor{codegray}{rgb}{0.5,0.5,0.5}
\definecolor{codepurple}{rgb}{0.58,0,0.82}
\definecolor{backcolour}{rgb}{0.95,0.95,0.92}

\lstdefinestyle{mystyle}{
    backgroundcolor=\color{backcolour},   
    commentstyle=\color{codegreen},
    keywordstyle=\color{magenta},
    numberstyle=\tiny\color{codegray},
    stringstyle=\color{codepurple},
    basicstyle=\ttfamily\footnotesize,
    breakatwhitespace=false,         
    breaklines=true,                 
    captionpos=b,                    
    keepspaces=true,                 
    numbers=left,                    
    numbersep=5pt,                  
    showspaces=false,                
    showstringspaces=false,
    showtabs=false,                  
    tabsize=2
}

\lstset{style=mystyle}


\title{}
\date{\today}
\author{202011010101物理2001孙陶庵}
\begin{document}
\begin{spacing}{2.0}
\tableofcontents
\maketitle
\section{研究背景(1-2页ppt)}
星系的重要性和磁场的作用:介绍星系作为宇宙中最大的天体结构之一的重要性,
以及磁场在星系形成和演化中的作用。解释磁场对星系内气体运动、星系结构、恒星形成等方面的影响。
星系磁场的观测历史:回顾星系磁场观测的历史发展,包括早期观测方法和设备的局限性,以及现代天文学技术的进步对磁场研究的影响。
提及星系磁场观测的主要手段,如测量同步辐射强度、偏振度、星光偏振等。
研究进展和发现:概述过去几十年内在星系磁场研究方面取得的重要进展和发现。包括发现不同类型的星系中存在磁场、
探测星系中磁场的强度和形态分布、磁场对星系结构和演化的影响等。
描述磁场在银河系和其他星系中的观测历史发展和研究进展
\subsection{磁场的起源(1-2页ppt)}
在星系中及其外部,磁场的起源、演化过程和空间特征仍是备受争议的。
磁场可以通过流体运动的感应效应从宇宙学显著小的值放大至当前星系水平。磁场能量的增加大致可以达到与流体湍流动能相等,
在星系中观测到的结果与之相符。然而,这幅图景存在一个严重问题。如果没有磁扩散作用(即,磁雷诺数Rem->00;见Box 2),
且磁场占据同一体积,则磁场强度的增加必须伴随着场线长度的比例增加,从而逐步使磁场缠绕。由于星际等离子体的欧姆电阻率极低,
预计星系中的Lm.s.磁场会远远超过平均场数个数量级,与它们的几乎相等形成鲜明对比。研究人员提出两种可行方式以解决这一困境。
其一,星际湍流能够将磁扩散性增大到欧姆值以上数个数量级(一种类比示例:在咖啡中搅动的奶油更快地变得光滑分布)。
其二,在紊流放大的图景中,相邻场线的抵消程度取决于流体流动的具体细节。磁场的性质与星际气体的动力学和能量学联系在一起。
磁场及其宇宙线对星系中的宇宙射线产生重要影响。磁场会影响星系的形成和早期演化,并在可观察的方面与恒星形成相互作用。
因此,我们预计磁场在塑造星际气体的全局结构方面发挥着重要作用,而气体状态反过来影响磁场。这一自洽态尚没有被观测和理论上的表征,
未来将面临巨大的挑战。

\section{理论基础(2-4页ppt,包含两个理论但不含公式介绍)}
介绍磁场在星系中的理论研究进展
讨论发电机理论和原初磁场理论作为解释磁场起源的两种主要观点
解释发电机理论中的磁场放大过程和原初磁场理论中的早期磁场产生
\subsection{发电机理论}
发电机理论是指在高磁导率的等离子体中通过流体运动产生磁场的过程。该过程中,磁场通过磁感线的运动产生感应电场,导致电子和离子的运动形成电流,
使得磁场得以维持。这个过程可以用麦克斯韦方程的简化形式来描述。星系的磁场由发电机理论和原初磁场理论两种解释方式,
前者是指通过大尺度流体运动和小尺度湍流运动产生磁场,后者则认为星系磁场源自于早期宇宙的强磁场。在研究星系磁场的同时,也探讨了磁场在宇宙其他领域的作用。
\subsection{原初磁场理论}
星系磁场来源可能是动力学漩涡发动机或原初场,前者会不断通过流体运动再生磁场,后者则是自行衰减的磁场在早期宇宙中残留下来。
论文指出,目前做出判断的两个不同理论将会提供对磁场起源和演化的观测测试,例如,竖直场的奇偶性和在银河系平面中的轴对称性和双轴对称性。
同样的,对于更远处很年轻的星系中是否存在磁场也是需要确定的问题
\section{测量方法(1-2页ppt)}
磁场能够被流体运动的感应效应所放大,然而这一过程存在困难,因为磁场的增强必然会伴随着磁场线的纠缠。
文章指出,这个难题可以通过增加磁扰动来解决。此外,磁场与气体动力学相互作用,有关星系中氢气柱密度的磁场强度已被测量,
并发现磁场的均方根值与均匀场强度的比值约为3。此外,该文还提到了星系早期磁场产生的可能性,以及磁场在形态上的涨落特征。
综上所述,磁场在原始气体中的塑造密度涨落是由流体运动的感应效应所导致的,磁场的增强可以通过增加磁扰动来解决。

该论文介绍了星系及其外部磁场的研究,包括通过测量同步辐射强度、偏振度和来自尘埃的红外排放等可观测量来探明磁场在原始气体中的作用。
研究证实,星系中的磁场对于星际气体动力学和星形成具有重要的影响。天文磁场的起源、性质与星际气体的动态和能量的关系密切相连,
磁场和宇宙线对气体提供极大的垂直支持。磁场与星际气体的相互作用的本质现在还没有观测上的明确特征,
因此,在未来的研究中,开发和测试这方面的模型将是一个巨大的挑战。

\section{研究方法(2-4页ppt)}
包括对非相对论体系下,基于麦克斯韦方程和电流密度的磁感应方程的求解,以及通过观测宇宙中的自然现象,
如旋转测量、射电磁波、吸收线和RM度量等等来间接探测磁场。此外,也讨论了宇宙中出现微高斯强度磁场的问题,
提出了传统平均场星系动力学理论难以解释的问题,引入了小尺度湍动动力学模型、强湍动放大和巨磁阻尼效应等理论模型来解释这些现象。
\section{研究成果(1-2页ppt)}
结果表明,在星系中存在着较大的磁场,而这些磁场可能是通过星系内动力学及星形成过程中的磁场放大机制形成的。
此外,该文还探讨了类星体和星系团中的磁场,并提出这些磁场可能是通过磁流体力学效应产生的。
在红移为z ~ 1的星系中发现了微高斯强度的磁场,这对于平均磁场星系动力学理论来说是个挑战。因此,磁场生成的过程需要进一步研究和限制。
\subsection{结论与展望(1页ppt)}
磁场对星系及其外部的物质动力学和宇宙射线等物理现象都有着显著影响,并对星系的形成和早期演化起着重要作用。银河系中的磁场起源至今仍存在争议。
现有的理论模型包括基于磁体热动力学效应的磁星的形成机制和动力学发生机制。目前支持这两种理论的观测结果仍不确定和不完整。
未来的观测研究需要使用现有和升级的射电天文望远镜等设备来挑战现有理论的限制。同时,还需通过观察更多银河系磁场与星旋结构关系较好的星系,
以探究星系螺旋结构、宇宙射线等的理论预测。
星系及其外部磁场理论的两个主要观点是发动机理论和原初理论。在发动机理论中,星系的磁扩散性较高,如果不通过流体运动不断再生磁场,
磁场会很快衰减。相比之下,原初理论认为星系的磁扩散性很低,现在星系中不会大量产生或破坏磁通量,现有的磁场是早期创造的。
发动机理论预测了星系介质磁场在大尺度流体运动(例如,星系旋转或星系气体流)和小尺度湍流流动作用下的平均磁场Bu,所涉及的尺度范围太大,
无法直接建模。因此,很多研究尝试通过运输系数来参数化湍流的影响,代表平均场的感应和破坏。最简单版本的平均场动力学理论中,
控制方程在Buo方面是线性的,如果电阻率足够小(但非零),磁场可以呈指数增长。许多模拟试图超越线性理论,考虑Bu施加的力量,抑制动力学作用,
这种“耗散”被认为将在平均场的能量达到湍流能量平衡时变得显著,因此Bu达到了其约等于平衡值的饱和状态,这与观测大致一致。
原初理论需要解释磁场预测的近似相等的平均场和随机场。因此,发动机理论和原初场理论的可行性在相当程度上取决于星系中的磁场扩散,
但在较极端的情况下,该问题所涉及的许多问题与描述星系磁场的问题相同。发现连续的星系间磁场将是非常重要的。这表明在星系形成之前就已存在磁化,
并且显示出星系和第一代恒星可能受到了磁场的影响。在更高角分辨率下观察更多的星系可能会发现更多的磁场反转。

\end{spacing}{}
\bibliographystyle{IEEEtran}
\bibliography{6thre}

\end{document}