\documentclass[12pt, a4paper, oneside]{article}
\usepackage{amsmath, amsthm, amssymb, graphicx}
\usepackage[bookmarks=true, colorlinks, citecolor=blue, linkcolor=black]{hyperref}
\usepackage[margin = 25mm]{geometry}
\usepackage{setspace}
\usepackage{listings}
\usepackage{ctex}
\usepackage{cite}
\usepackage{color}


\title{设计一个长距离光纤传输系统}
\date{\today}
\author{Alphabetium}
\begin{document}
\begin{spacing}{2.0}
\tableofcontents
\maketitle


\section{背景介绍}
在现代通信领域,光纤传输系统扮演着至关重要的角色,为高速、大容量的数据传输提供了可靠的解决方案。随着通信需求的不断增长,
设计和优化高性能光纤传输系统变得越来越重要。本报告旨在设计一个4×2.5 Gbit/s的长距离光纤传输系统,并利用OptiSystem仿真验证其性能。
\\
任务要求:
本次任务要求采用波分复用技术,通过波分复用器将不同波长的光信号合并传输。每个波长的传输速率为2.5 Gbit/s(NRZ),
波长间隔为100 GHz,频率分别为193.1 THz、193.2 THz、193.3 THz和193.4 THz。传输距离为300 km,并使用常规光纤,其光纤损耗系数为0.25 dB/km。
\\
设计要求:
为了满足任务要求,我们将采用EDFA作为线路放大器,并根据其工作参数进行设计。EDFA的输入功率限制在-18 dBm至2 dBm之间,最大输出功率为20 dBm,
增益范围为13 dB至25 dB,噪声指数为4.0 dB。系统设计时,我们仅考虑了损耗要求,以确保信号传输的稳定性和质量。
\\
仿真验证:
为了验证所设计的传输系统的性能,我们将利用OptiSystem进行仿真试验。通过仿真,我们将优化设计方案,以使系统接收端的输入功率和光信号的光信噪比(OSNR)达到最优。
通过仿真实验,我们可以评估系统的性能并提出改进措施,从而为实际系统的设计和部署提供有价值的参考。
\\
通过本报告的研究和分析,我们将深入了解光纤传输系统的设计原理和性能优化方法,为高速、长距离数据传输提供有效的解决方案。
这对于满足日益增长的通信需求和推动信息技术的发展具有重要意义。

\section{系统设计}
\subsection{光纤传输系统概述}
1.传输速率和通道配置\\
2.波长选择和波分复用技术\\
3.光纤损耗系数和传输距离\\
\subsection{光源和光接收机设计}
1.光源功率和最大输出功率要求\\
2.光接收机灵敏度和过载点要求\\


\subsection{系统设计图}
1.示意图\\
2.光学组件配置(截图)



\section{OptiSystem仿真}
1.仿真参数和系统配置\\
2.光纤传输和放大器设置


\subsection{仿真结果分析}
输入功率和OSNR分析

\section{结论2}

\end{spacing}{}
\bibliographystyle{IEEEtran}
\bibliography{re1}

\end{document}