\documentclass[12pt, a4paper, oneside]{ctexart}
\usepackage{amsmath, amsthm, amssymb, graphicx}
\usepackage[bookmarks=true, colorlinks, citecolor=blue, linkcolor=black]{hyperref}
\usepackage[margin = 25mm]{geometry}
\usepackage{setspace}
\usepackage{listings}
\usepackage{cite}
\usepackage{ctex}
\usepackage{float}
\usepackage{xcolor}

\definecolor{codegreen}{rgb}{0,0.6,0}
\definecolor{codegray}{rgb}{0.5,0.5,0.5}
\definecolor{codepurple}{rgb}{0.58,0,0.82}
\definecolor{backcolour}{rgb}{0.95,0.95,0.92}

\lstdefinestyle{mystyle}{
    backgroundcolor=\color{backcolour},   
    commentstyle=\color{codegreen},
    keywordstyle=\color{magenta},
    numberstyle=\tiny\color{codegray},
    stringstyle=\color{codepurple},
    basicstyle=\ttfamily\footnotesize,
    breakatwhitespace=false,         
    breaklines=true,                 
    captionpos=b,                    
    keepspaces=true,                 
    numbers=left,                    
    numbersep=5pt,                  
    showspaces=false,                
    showstringspaces=false,
    showtabs=false,                  
    tabsize=2
}

\lstset{style=mystyle}


\title{WKB近似的介绍讲解以及它在不同领域的应用}
\date{\today}
\author{202011010101物理2001孙陶庵}
\begin{document}
\begin{spacing}{2.0}
\tableofcontents
\maketitle

\section{引言}
量子物理中的WKB(Wentzel-Kramers-Brillouin)近似是一种强大的工具,用于解决薛定谔方程的近似解。
它在描述均匀介质中的量子现象和波动行为方面具有广泛的应用。然而,许多实际系统,如非均匀等离子体,具有空间上的变化性质,
导致了更为复杂的物理行为。在这种情况下,我们需要将WKB近似扩展到非均匀介质,以更准确地描述系统的性质和行为。

非均匀等离子体作为一种重要的物质状态,在激光等离子体物理和实验室天体物理中扮演着关键的角色。
它们在实验室中通过激光和高能粒子束的相互作用中产生,并在宇宙中存在于星际介质和恒星大气等环境中。
然而,由于等离子体密度、温度和电场等物理量在空间上的变化,非均匀等离子体的性质和行为比均匀介质更为复杂和多样化。
WKB方法在等离子体物理中是非常有价值的解决方案,因为这个方法提供了一个直观的方式描述波的传播;
通过一个局部色散关系定义的波,其幅度和相位以一种非常明显的方式与这种局部的特征相关

在这篇文章中,我们会讨论WKB近似在非均匀等离子体中的应用。首先,我们将回顾WKB近似的基本原理和推导过程。
然后在激光等离子体中,我们将探讨WKB近似在分析波模式中的应用。

\section{WKB近似的基本原理}
WKB(Wenzel,Kramers, Brillouin)方法是得到一维定态 Schrödinger 方程的近似解的一种技术,
其基本思想同样可应用于许多其他形式的微分方程和三维 Schrödinger 方程的径向部分,其最基本的核心思想主要是:
首先波函数以指数函数的形式重新表达,再将这指数函数代入Schrödinger方程,展开指数函数的参数为$\hbar$的幂级数,
$\hbar$同次幂的项目一一对应,会得到一组方程,处理后,就会得到波函数的近似。
\subsection{基本思想}
中间省略了很多步骤,详细过程参见各大量子力学课本\cite{griffiths_schroeter_2018},以及这篇文章皆有非常好的讲解\cite{LU2018502}
对于一维定态Schrödinger方程:
\begin{center}
    $\displaystyle - \frac{\hbar^2}{2m} \frac{\mathrm{d}^2}{\mathrm{d}x^2} \psi(x) + V(x) \psi(x) = E \psi(x)\,\!$
\end{center}
将其重写为:
\begin{center}
    $\displaystyle -\hbar^2\frac{\mathrm{d}^2}{\mathrm{d}x^2} \psi(x) =p^2\psi(x)\,\!$
\end{center}
其中$p(x)\equiv \sqrt{2m[E-V(x)]}$,此时假设$E>V(x)$,因此p为实数,此为经典区域,所以现在假设波函数的形式为另外一个函数$\phi\,\!$的指数,
$\displaystyle \psi(x) = e^{\phi(x)/\hbar} \,\!$。将其代回原方程可以得到
\begin{center}
    $\displaystyle\frac{d\psi}{dx}=\left(A^\cdot+i A\phi^\cdot\right)e^{i\phi}$
\end{center}
\begin{center}
    $\displaystyle\frac{d^2\psi}{dx^2}=\bigg[A'+2i A\phi'+i A\phi'-A\Big(\phi'\Big)^2\bigg]e^{i\phi}$
\end{center}
代回原式可得:
\begin{center}
    $\displaystyle A^{'}+2i A^{'}\phi+i A\phi^{'}-A\Big(\phi^{'}\Big)^{2}=-\frac{p^{2}}{\hbar^{2}}A$
\end{center}
这等价于两个实数方程,且一个实部一个虚部:
\begin{center}
    $\displaystyle A^*-A\Big(\phi^.\Big)^2=-\frac{p^2}{\hbar^2}A$
\end{center}
\begin{center}
    $2A\phi'+A\phi'=0$
\end{center}
第二个方程很容易解出:
\begin{center}
    $A^2\phi^2=C^2$
\end{center}
式中 C 为(实)常数。一般来说第一个方程很难求解⎯所以需要近似:我们假定振幅 A 的变
化非常缓慢,因此$A''$项可忽略,在此情况下,我们只剩下:
\begin{center}
    $\displaystyle\phi(x) = \pm \int p(x) \mathrm{dx}$
\end{center}
可以得出:
\begin{center}
    $\displaystyle\psi(x) \approx  \frac{C_{\pm}} {\sqrt{p(x)}}  e^{\frac{i}{\hbar}\pm\int p(x) \mathrm{d}x}$
\end{center}
接着,在势阱内部,我们有:
\begin{center}
    $\displaystyle\psi\big(x\big)\cong\frac{1}{\sqrt{p(x)}}\biggl[C_+e^{i\phi(x)}+C_-e^{-i\phi(x)}\biggr]$
\end{center}
其中
\begin{center}
    $\displaystyle \phi(x) = \frac{1}{\hbar}\int_0^xp(x')\mathrm{dx'}$
\end{center}
现在考虑边界条件$\psi(x=0)=0$和$\psi(x=a)=0$,所以$C_2=0$,所以
\begin{center}
    $\phi\bigl(a\bigr)=n\pi\quad\bigl(n=1,2,3,...\bigr)$
\end{center}
最后,系统满足量子化规则:
\begin{center}
    $\displaystyle \int_{x_1}^{x_2} p(x)dx =(n - 1/2)\pi\hbar,\qquad n=1,\,2,\,3,\,\dots\,\!$
\end{center}

\subsection{解释和物理意义}
1.上面讲的这些主要是WKB近似的基本定义(做法),这个方法不仅可以用来解决量子力学的问题,对于绝大部分类薛定谔方程(Schrodinger-like differential equation)都能够很好的求解\\
2.WKB近似法能够在不需要精确求解的情况下,通过计算得到等离子体中波的传播方程,并得出物理运动的定性描述。
WKB近似法的优势在于它比其他精确方法更快和更简单,并且可以解决一些无法通过其他方法计算的难点问题。
\\
3.带有标量场的封闭弗里德曼宇宙的变形Wheeler–DeWitt方程的WKB解,并在隧道建议的背景下讨论了量子引力对充分膨胀概率的影响。\cite{LU2018502}

\section{WKB近似在激光等离子体中的应用}
WKB近似描述波的传播较为直观,可以通过局域色散关系定义一个波的波数、群速度、振幅、相位等
对于线性密度变化的等离子体,可以得到精确解,但WKB方法仍然有其局限性,即仅在缓慢的密度变化成立,但临界点不成立
\subsection{非均匀等离子体中波的传播}
一般来说等离子体非均匀密度问题需要数值计算,但是在特殊情况可以有解析解:
1.密度分布缓变,$kL\gg 1$,使用WKB近似可以分析,不过在共振点时会失效\cite{lasertextbook11223}
\\
2.密度分布剧变,$kL\ll 1$,即表面鞘场
\subsubsection{过程介绍}
我们接下来考虑下面形式的高频场:
\begin{center}
    $\displaystyle E = E(\vec{x})\exp (-\mathrm{i\omega t})$
\end{center}
由波动方程$\displaystyle \nabla^{2}E-\nabla(\nabla\cdot E)=\frac{4\pi}{c^{2}}\frac{\partial J}{\partial t}+\frac{1}{c^{2}}\frac{\partial^{2}E}{\partial t}$和$\displaystyle J = \frac{ie^2n}{\omega}E = \sigma E$
,$\displaystyle \varepsilon\equiv 1+\frac{4\pi i \sigma}{\omega}$,可以得到

\begin{center}
    $\displaystyle \nabla^{2}E-\nabla(\nabla\cdot E)+{\frac{\omega^{2}}{c^{2}}}\varepsilon E=0$
\end{center}
\begin{center}
    $\displaystyle \nabla\times(\nabla\times B)=-{\frac{-i\omega}{c}}\nabla\times(\varepsilon E)$
\end{center}
后者可以得到
\begin{center}
    $\displaystyle \nabla^2 B + \frac{\omega^2}{c^2}\varepsilon B + \frac{1}{\varepsilon}\nabla \varepsilon \times(\nabla \times B) = 0$
\end{center}
电场的空间形式$\displaystyle \exp(ik\cdot \vec{x})$可以给出色散关系:$\displaystyle \omega^2 = \omega_{pe}^2+k^2c^2$
接着我们假定等离子体的密度只在Z方向上有变化,对于线偏振激光电场E满足以下关系
\begin{center}
    $\displaystyle\frac{d^{2}E}{d z^{2}}+k^{2}(z)E=0$
\end{center}
\begin{center}
    $\displaystyle k^{2}(z)=\frac{\omega^{2}-\omega_{p}^{2}(z)}{c^{2}}=\frac{\omega^{2}}{c^{2}}\varepsilon$
\end{center}
并且假定等离子体密度对Z是缓变的,这样可以给出一个形式解
\begin{center}
    $\displaystyle E(z)=E_0(z)\exp[\frac{i\omega}{c}\int\psi(z')dz']$
\end{center}
因为这里$E_0, \psi$是缓变函数,将其带入波动方程会得到
\begin{center}
    $\displaystyle E_{0}^{''}+\frac{2i\omega}{c}\psi E_{0}^{'}-\frac{\omega^{2}}{c^{2}}\psi^{2}E_{0}+\frac{i\omega}{c}\psi^{'}E_{0}+\frac{\omega^{2}}{c^{2}}\varepsilon E_{0}=0$
\end{center}
方程有零阶和一阶解,对于零阶有$\displaystyle \psi = \sqrt{\varepsilon(\omega, z)}$
对于一阶有$\displaystyle \frac{2i\omega}{c}\psi E_0+\frac{i\omega\psi'E_0}{c}=0$可以由此得到$\displaystyle E_0(z)=\frac{cons\tan t}{\sqrt{\psi}}$
最终获得WKB解:
\begin{center}
    $\displaystyle E(z)=\frac{E_{F S}}{\varepsilon^{1/4}}\exp[\frac{i\omega}{c}\int\sqrt{\varepsilon(\omega,z')}d z']$
\end{center}
1.从能流守恒解释:
\\
$\displaystyle \frac{\nu_{g}\mid E(z)\mid^{2}}{8\pi}=\frac{c E_{F S}^{2}}{8\pi},\quad\nu_{g}/c=\sqrt{\varepsilon(\omega,z)}$
因此:$\displaystyle E(z) = \frac{E_{FS}}{\varepsilon^{1/4}}$与WKB结果吻合,也可写为磁场形式:$\displaystyle |B_0(z)| = B_{FS}\varepsilon^{1/4}$磁场振幅随激光向高密度传播降低

2.讨论WKB的适用条件:
\\
我们在前面的推导中假定:$\displaystyle E_{0}<<\frac{\omega}{c}\psi"E_{0},\frac{\omega}{c}\psi E_{0}$
从$\displaystyle E(z)=\frac{E_{F S}}{\varepsilon^{1/4}}\mathbf{exp}[\frac{i\omega}{c}^{z}]\sqrt{\varepsilon(\omega,z^{\prime})}d z^{\prime}$
可以得到$\mathbf{k}(z)=\omega\psi/\mathbf{c}$,有:$E_0''\ll k'E_0,kE'_0$
因此可以知道WKB近似的适用条件:$\displaystyle \frac{\partial E_0}{\partial z}\ll kE_0$


\subsection{波在固定密度梯度等离子体的解析解}
我们接下来考虑一个线性密度的等离子体$(n = n_{cr}z/L)$其中$n_{cr} = m\omega^2/4\pi e^2$\cite{lasertextbook1122}
可以得到$\displaystyle \frac{\mathrm{d}^2E}{\mathrm{d}z^2}+\frac{\omega^2}{c^2}\left(1-\frac{z}{L}\right)E=0.$
我们令$\eta = (\omega^2/c^2L)^{1/3}(z-L)$有$\displaystyle \frac{\mathrm{d}^2E}{\mathrm{d}\eta^2}-\eta E=0.$
这是艾里方程,其解称为艾里函数。因为艾里方程是一个二阶微分方程,所以有两个线性独立的艾里函数,Ai(z)和 Bi(z)(见图)。
\begin{figure}
    \centering
	\includegraphics[width=10cm]{airyfunction.jpg}
	\caption{Airy Function}
\end{figure}

对波函数在转折点(E=V)处进行处理, “经典”区和“非经典”区在此处相接
透过处理这个方程可以(过程略,详细可参见\cite{griffiths_schroeter_2018}的8.3 THE CONNECTION FORMULAS)將 $E(z = 0)$ 表示為振幅為$E_{FS}$的入射波和具有相同振幅但相位偏移的反射波之和,即,
\begin{center}
    $\displaystyle E(z=0)=E_{\text{Fs}}\left[1+\exp-i\left(\frac{4}{3}\frac{\omega L}{c}-\frac{\pi}{2}\right)\right],$
\end{center}
这里$E_{FS}$是入射光波电场的自由空间的值,且$\varphi$只是一个相位因子且无法影响$|E|$,因此,
\begin{center}
    $\displaystyle E(\eta)=2\sqrt{\pi}\left(\frac{\omega L}{c}\right)^{1/6}E_{\text{FS}}e^{i\varphi}\mathbf{A}_i(\eta).$
\end{center}
图中可以看出电场的振幅在$\eta = 1$时可以达到最大值,即$(z-L) = -(\omega^2/c^2L)^{-1/3}$
\begin{center}
    $\displaystyle \left|\frac{E_{\max}}{E_{\text{FS}}}\right|^2\simeq3.6\left(\frac{\omega L}{c}\right)^{1/3}.$
\end{center}
因为驻波的形成,$E^2$的期望值增大四倍,这是因为介电函数变小和群速度变小。
接着,可以基于WKB理论获得峰值电场幅度的类似膨胀。 这里我们使用$k = \sqrt{\epsilon}(\omega/c)$和$|E| = E_{FS}/\epsilon^{1/4}$
随着$\epsilon$变小,波长变长。 
同样的,如果我们简单地减去$\pi/2$以说明临界密度表面的反射,则入射波和反射波之间的相移由 WKB 解给出。即
\begin{center}
    $\displaystyle \Psi=2\frac{\omega}{c}\int_{0}^{L}\sqrt{\epsilon}\mathrm{d}z-\frac{\pi}{2}=\frac{4}{3}\frac{\omega L}{c}-\frac{\pi}{2}.$
\end{center}

光波的磁场很容易从 E 的解中计算出来。注意到电矢量在 x 方向上并且波在 z 方向上传播,我们采用法拉第定律的 y 分量来获得
\begin{center}
    $\displaystyle B=-\frac{ic}{\omega}\frac{\partial E}{\partial z}.$
\end{center}即
\begin{center}
    $\displaystyle B(\eta)=-i2\sqrt{\pi}\left(\frac{c}{\omega L}\right)^{1/6}E_{\text{FS}}e^{i\varphi}\mathrm{A}_i'(\eta),$
\end{center}

这是一个非常棒的例子,不仅严格的按照WKB近似的步骤,并且非常合理的说明了波在等离子体中传播的问题。

\section{结论}
综上所述,WKB(Wentzel-Kramers-Brillouin)近似是一种强大的工具,用于解决量子力学中的薛定谔方程的近似解。
在均匀介质中,WKB近似已经得到广泛的应用,能够描述量子现象和波动行为。
WKB近似相对于其他精确的方法,WKB近似更简单、更直观,并且可以解决一些难以通过其他方法计算的问题。
然而,许多实际系统,特别是非均匀等离子体,具有空间上的变化性质,会需要解决更为复杂的物理行为。
通过WKB近似,我们可以得到非均匀等离子体中波的传播方程,并对物理运动进行定性描述。

并且,WKB近似可以应用于解决类薛定谔的偏微分方程的近似解。将WKB近似扩展到非均匀介质,
特别是在非均匀等离子体中的应用,可以更准确地描述系统的性质和行为。尽管WKB近似在一些情况下存在局限性,
但它仍然是研究非均匀等离子体中波动行为的非常有用的方法。








\end{spacing}{}

\bibliographystyle{IEEEtran}
\bibliography{qp}

\end{document}