\documentclass[12pt, a4paper, oneside]{ctexart}
\usepackage{amsmath, amsthm, amssymb, graphicx}
\usepackage[bookmarks=true, colorlinks, citecolor=blue, linkcolor=black]{hyperref}
\usepackage[margin = 25mm]{geometry}
\usepackage{setspace}
\usepackage{listings}
\usepackage{cite}
\usepackage{ctex}
\usepackage{float}
\usepackage{xcolor}

\definecolor{codegreen}{rgb}{0,0.6,0}
\definecolor{codegray}{rgb}{0.5,0.5,0.5}
\definecolor{codepurple}{rgb}{0.58,0,0.82}
\definecolor{backcolour}{rgb}{0.95,0.95,0.92}

\lstdefinestyle{mystyle}{
    backgroundcolor=\color{backcolour},   
    commentstyle=\color{codegreen},
    keywordstyle=\color{magenta},
    numberstyle=\tiny\color{codegray},
    stringstyle=\color{codepurple},
    basicstyle=\ttfamily\footnotesize,
    breakatwhitespace=false,         
    breaklines=true,                 
    captionpos=b,                    
    keepspaces=true,                 
    numbers=left,                    
    numbersep=5pt,                  
    showspaces=false,                
    showstringspaces=false,
    showtabs=false,                  
    tabsize=2
}

\lstset{style=mystyle}


\title{}
\date{\today}
\author{202011010101物理2001孙陶庵}
\begin{document}
\begin{spacing}{2.0}
\tableofcontents
\maketitle

\section{引言}
量子物理中的WKB(Wentzel-Kramers-Brillouin)近似是一种强大的工具,用于解决薛定谔方程的近似解。
它在描述均匀介质中的量子现象和波动行为方面具有广泛的应用。然而,许多实际系统,如非均匀等离子体,具有空间上的变化性质,
导致了更为复杂的物理行为。在这种情况下,我们需要将WKB近似扩展到非均匀介质,以更准确地描述系统的性质和行为。

非均匀等离子体作为一种重要的物质状态,在激光等离子体物理和实验室天体物理中扮演着关键的角色。
它们在实验室中通过激光和高能粒子束的相互作用中产生,并在宇宙中存在于星际介质和恒星大气等环境中。
然而,由于等离子体密度、温度和电场等物理量在空间上的变化,非均匀等离子体的性质和行为比均匀介质更为复杂和多样化。

在这篇论文中,我们将探索WKB近似在非均匀等离子体中的应用。首先,我们将回顾WKB近似的基本原理和推导过程,以及其在均匀介质中的应用。
然后在激光等离子体中,我们将探讨WKB近似在分析波模式、自聚焦效应和光束传输中的应用。

\section{WKB近似的基本原理}
WKB(Wenzel,Kramers, Brillouin)方法是得到一维定态 Schrödinger 方程的近似解的一种技术,
其基本思想同样可应用于许多其他形式的微分方程和三维 Schrödinger 方程的径向部分,其最基本的核心思想主要是:
首先波函数以指数函数的形式重新表达,再将这指数函数代入Schrödinger方程,展开指数函数的参数为$\hbar$的幂级数,
$\hbar$同次幂的项目一一对应,会得到一组方程,处理后,就会得到波函数的近似。
\subsection{基本思想}
中间省略了很多步骤,详细过程参见各大量子力学课本\cite{griffiths_schroeter_2018}
对于一维定态Schrödinger方程:
\begin{center}
    $\displaystyle - \frac{\hbar^2}{2m} \frac{\mathrm{d}^2}{\mathrm{d}x^2} \psi(x) + V(x) \psi(x) = E \psi(x)\,\!$
\end{center}
将其重写为:
\begin{center}
    $\displaystyle -\hbar^2\frac{\mathrm{d}^2}{\mathrm{d}x^2} \psi(x) =p^2\psi(x)\,\!$
\end{center}
其中$p(x)\equiv \sqrt{2m[E-V(x)]}$,此时假设$E>V(x)$,因此p为实数,此为经典区域,所以现在假设波函数的形式为另外一个函数$\phi\,\!$的指数,
$\displaystyle \psi(x) = e^{\phi(x)/\hbar} \,\!$。将其代回原方程可以得到
\begin{center}
    $\displaystyle\frac{d\psi}{dx}=\left(A^\cdot+i A\phi^\cdot\right)e^{i\phi}$
\end{center}
\begin{center}
    $\displaystyle\frac{d^2\psi}{dx^2}=\bigg[A'+2i A\phi'+i A\phi'-A\Big(\phi'\Big)^2\bigg]e^{i\phi}$
\end{center}
代回原式可得:
\begin{center}
    $\displaystyle A^{'}+2i A^{'}\phi+i A\phi^{'}-A\Big(\phi^{'}\Big)^{2}=-\frac{p^{2}}{\hbar^{2}}A$
\end{center}
这等价于两个实数方程,且一个实部一个虚部:
\begin{center}
    $\displaystyle A^*-A\Big(\phi^.\Big)^2=-\frac{p^2}{\hbar^2}A$
\end{center}
\begin{center}
    $2A\phi'+A\phi'=0$
\end{center}
第二个方程很容易解出:
\begin{center}
    $A^2\phi^2=C^2$
\end{center}
式中 C 为(实)常数。一般来说第一个方程很难求解⎯所以需要近似:我们假定振幅 A 的变
化非常缓慢,因此$A''$项可忽略,在此情况下,我们只剩下:
\begin{center}
    $\displaystyle\phi(x) = \pm \int p(x) \mathrm{dx}$
\end{center}
可以得出:
\begin{center}
    $\displaystyle\psi(x) \approx  \frac{C_{\pm}} {\sqrt{p(x)}}  e^{\frac{i}{\hbar}\pm\int p(x) \mathrm{d}x}$
\end{center}
接着,在势阱内部,我们有:
\begin{center}
    $\displaystyle\psi\big(x\big)\cong\frac{1}{\sqrt{p(x)}}\biggl[C_+e^{i\phi(x)}+C_-e^{-i\phi(x)}\biggr]$
\end{center}
其中
\begin{center}
    $\displaystyle \phi(x) = \frac{1}{\hbar}\int_0^xp(x')\mathrm{dx'}$
\end{center}
现在考虑边界条件$\psi(x=0)=0$和$\psi(x=a)=0$,所以$C_2=0$,所以
\begin{center}
    $\phi\bigl(a\bigr)=n\pi\quad\bigl(n=1,2,3,...\bigr)$
\end{center}
最后,系统满足量子化规则:
\begin{center}
    $\displaystyle \int_{x_1}^{x_2} p(x)dx =(n - 1/2)\pi\hbar,\qquad n=1,\,2,\,3,\,\dots\,\!$
\end{center}

\subsection{解释和物理意义}
1.隧穿效应:WKB近似在描述隧穿效应(tunneling effect)方面非常有用。当粒子遇到势垒或势阱时,根据经典物理学,粒子应该被完全反射或完全传播。
然而,量子力学中存在隧穿效应,使得一部分波函数能够穿透势垒或势阱。WKB近似能够提供隧穿概率的估计,从而解释一些实验现象,如$\alpha$衰变和扫描隧道显微镜等。
\\
2.粒子轨道和量子态:WKB近似还可以用于计算量子力学中的粒子轨道和量子态。
在某些情况下,例如静电场中的粒子运动或氢原子的束缚态,WKB近似可以提供近似的能级和波函数。
\\
3.WKB近似法能够在不需要精确求解的情况下,通过计算得到等离子体中波的传播方程,并得出物理运动的定性描述。
WKB近似法的优势在于它比其他精确方法更快和更简单,并且可以解决一些无法通过其他方法计算的难点问题。
\section{WKB近似在激光等离子体中的应用}
接下来考虑在背景密度和磁场随位置但不随时间变化的情况下,略微不均匀的冷等离子体中的波传播,并且揭示等离子体中的波传播特性。
\subsection{不均匀等离子体中波的传播机制}
1.折射:当波从一个介质传播到另一个介质时,波会受到折射现象的影响。在不均匀冷等离子体中,由于等离子体的密度或折射率分布不均匀,
波在介质中传播时会遇到密度或折射率的变化。这会导致波的传播方向发生改变,遵循折射定律。折射现象在不均匀冷等离子体中是波传播的重要机制之一。
\\
2.衍射:衍射是波在通过障碍物或通过波前上的不规则结构时发生的现象。在不均匀冷等离子体中,波通过介质中的非均匀性分布时,
会遇到空间上的不规则结构,如孔隙、缺陷或波动性变化。这些不规则结构会导致波的传播方向和强度的变化,产生衍射现象。
\\


\subsection{冷等离子体波、WKB近似和射线追踪的应用}
在冷等离子波笔记中,我们假设波的电磁场具有以下形式:
\begin{center}
    $\displaystyle \delta\mathbf{E}=\tilde{\mathbf{E}}\exp(\text{i}\mathbf{k}\cdot\mathbf{r}-\text{i}\omega t),$
\end{center}
\begin{center}
    $\displaystyle \delta\mathbf{B}=\tilde{\mathbf{B}}\exp(\text{i}\mathbf{k}\cdot\mathbf{r}-\text{i}\omega t).$
\end{center}
在非均匀的稳态等离子体中,我们不能简单地假设波函数是$\displaystyle \exp(\text{i}\mathbf{k}\cdot\mathbf{r}-\text{i}\omega t)$,
而要考虑另一种形式,即
\begin{center}
    $\displaystyle\delta\mathbf{E}=\tilde{\mathbf{E}}(\mathbf{r})\exp(\mathrm{i}S(\mathbf{r})-\mathrm{i}\omega t),$
\end{center}
\begin{center}
    $\displaystyle \delta\mathbf{B}=\tilde{\mathbf{B}}(\mathbf{r})\exp(\mathrm{i}S(\mathbf{r})-\mathrm{i}\omega t).$
\end{center}
我们假设 S为 eikonal 函数\cite{DETRIXHE201346}并且系数E和B具有L量级的特征长度,

\begin{center}
    $\displaystyle \frac{|\nabla S|}{S}\sim\frac{|\nabla\tilde{\mathbf{E}}|}{|\tilde{\mathbf{E}}|}\sim\frac{|\nabla\tilde{\mathbf{B}}|}{|\tilde{\mathbf{B}}|}\sim\frac{1}{L},$
\end{center}
同时方程S(r)很大

\begin{center}
    $\displaystyle S(\mathbf{r})\sim kL \gg  1$
\end{center}
,所以考虑
\begin{center}
    $\displaystyle |\nabla S|\sim k\gg\frac{1}{L}$
\end{center}
这样就可以重写冷等离子体方程,将$\displaystyle \mathrm{ik}\tilde{\mathbf{E}}$变成$\mathrm{i}\nabla S\tilde{\mathbf{E}}+\nabla\tilde{\mathbf{E}}$
我们就可以得到:
\begin{center}
    $\displaystyle\frac{c^2}{\omega^2}(\mathbf{k}-\mathrm i\nabla)\times[(\mathbf k-\mathrm i\nabla)\times\tilde{\mathbf E}]+\epsilon\cdot\tilde{\mathbf E}=0,$
\end{center}
其中$K = \nabla S$




\subsection{求得解析解}
\subsection{}



\section{结论}



\end{spacing}{}

\bibliographystyle{IEEEtran}
\bibliography{qp}

\end{document}