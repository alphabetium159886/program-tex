\documentclass[12pt, a4paper, oneside]{article}
\usepackage{amsmath, amsthm, amssymb, graphicx}
\usepackage[bookmarks=true, colorlinks, citecolor=blue, linkcolor=black]{hyperref}
\usepackage[margin = 25mm]{geometry}
\usepackage{setspace}
\usepackage{listings}
\usepackage{ctex}
\usepackage{cite}
\title{Title}
\date{\today}
\author{Alphabetium}
\begin{document}
\begin{spacing}{2.0}
\tableofcontents
\maketitle


\section{磁重聯简述}

磁重聯是發生在等離子體中的一個過程,最初分開的磁力線聚集在一起並合併,並在此過程中釋放能量。

磁重聯的原理可以用磁力線的概念來解釋。 磁場線就像想像中的線一樣,追踪等離子體中磁場的路徑。 它們有一個方向,可以被認為是從空間中的一點延伸到另一點。

當等離子體不同區域的磁力線聚集在一起時,它們可以重新連接,從而形成新的磁力線。 這個過程釋放能量並改變等離子體中磁場的拓撲結構。 
新的磁場線可以有不同的形狀和方向,這會導致等離子體動力學發生變化。

磁重聯是一個尚未完全理解的複雜過程,但它被認為在許多天體物理現像中起著至關重要的作用,例如太陽耀斑、日冕物質拋射和地球磁層。 
它還與實驗室等離子體、聚變能和空間天氣研究有關。
\section{磁重联的物理机制和基本模型}
\subsection{Sweet-Parker模型}
Sweet-Parker模型描述了在电阻磁流体力学框架下,当重新连接的磁场是反平行(方向相反)且与粘性和可压缩性相关的效应不重要时,
时间无关的磁再连接。初始速度仅为$E\times B$速度,因此:
\begin{center}
    $\displaystyle E_y = v_\text{in} B_\text{in}$
\end{center}
其中$E_y$是垂直于平面的电场,$v_\text{in}$是特征入流速度,而$B_\text{in}$则是特征上游磁场强度。通过忽略位移电流,可以得到低频安培定律。
$\displaystyle\mathbf{J} = \frac{1}{\mu_0}\nabla\times\mathbf{B}$, gives the relation: 
\begin{center}
    $\displaystyle J_y \sim \frac{B_\text{in}}{\mu_0\delta}$
\end{center}
其中$\delta$是电流层的半厚度。该关系使用磁场在约$2\delta$的距离内反转。通过将理想电场与层内阻性电场$\displaystyle\mathbf{E} = \frac{1}{\sigma}\mathbf{J}$(使用欧姆定律)相匹配,我们发现: 
\begin{center}
    $\displaystyle v_\text{in} = \frac{E_y}{B_\text{in}} \sim \frac{1}{\mu_0\sigma\delta} = \frac{\eta}{\delta}$
\end{center}
这里,$\eta$是磁扩散率。当入流密度与出流密度相当时,质量守恒产生以下关系:\begin{center}
    $\displaystyle v_\text{in}L \sim v_\text{out}\delta$
\end{center}
其中$L$是电流层的半长度,$v_\text{out}$是出流速度。上述关系式的左右两侧分别表示进入层和离开层的质量通量。
将上游磁压力与下游动压力相等:\begin{center}
    $\displaystyle \frac{B_\text{in}^2}{2\mu_0} \sim \frac{\rho v_\text{out}^2}{2}$
\end{center}
where $\rho$ is the mass density of the plasma.  Solving for the outflow velocity then gives: 
\begin{center}
    $\displaystyle v_\text{out} \sim \frac{B_\text{in}}{\sqrt{\mu_0\rho}} \equiv v_A$
\end{center}
其中 $v_A$ 是Alfvén wave波or Alfvén velocity。有了上述关系,无量纲重联速率 $R$ 就可以用两种形式来表示,
第一种是使用从欧姆定律中推导出的结果 $(\eta, \delta, v_A)$,第二种是使用质量守恒定律中的 $(\delta, L)$:
\begin{center}
    $\displaystyle R = \frac{v_\text{in}}{v_\text{out}} \sim \frac{\eta}{v_A\delta} \sim \frac{\delta}{L}$
\end{center}
Since the dimensionless Lundquist number $S$ is given by
: 
\begin{center}
    $\displaystyle S \equiv \frac{Lv_A}{\eta}$
\end{center}
$R$的两个不同表达式相乘,然后开方,给出了重新连接速率$R$和Lundquist数$S$之间的简单关系:
\begin{center}
    $\displaystyle R ~ \sim \sqrt{\frac{\eta}{v_A L}} = \frac{1}{S^\frac{1}{2}}$
\end{center}
Sweet-Parker 模型是一種加速機制,在 X 線附近,粒子可以被平行於重新連接的電場自由加速
平面。 如果重聯區延伸距離D,則最大可能的能量增益為 ZeED ,它可能很大,但大多數粒子在達到該能量之前會偏離 X 線。 
這種機制產生的粒子能譜不能很好地符合天體物理光譜,這降低了人們對這種機制的興趣\cite{zweibel2009magnetic}

\subsection{Petschek模型}
Sweet-Parker模型是比较慢的,因为所有进入重新连接区域的流体都必须通过一个有电阻的通道流出。
Petschek解释如果电阻层很短,大部分进入的流体不通过它流出,而是被静态激波重新定向,重新连接速度会更快。
Petschek的理论被广泛引用来支持快速重新连接,最大重新连接速度是$vA(π/8 \ln S)$,通常是Alfvén velocity 的百分之几,足以解释大多数天体物理现象。
但是,通过数值MHD模拟尝试验证Petschek的理论表明,这种重新连接不会自行发展,除非磁扩散率$\eta$在X点附近增加。
Petschek的模型的一个普遍特点是,与Sweet-Parker模型相比,大部分能量转化为流出的离子动能和热(如果存在激波),相对较少的能量进入电子的电阻加热中。

\subsection{Spontaneous Reconnection}
Sweet-Parker模型和Petschek模型描述了稳态重联,但没有解释它发生的情况。Furth、Killeen和Rosenbluth(1963)引入了一个新概念,

他们表明磁场可以对小扰动(称为tearing mode)不稳定,并重新连接磁力线。
图3显示了tearing mode的示例。在图的平面方向上有一个强大而几乎均匀的场,
称为引导场。因为平面内的By在$x = 0$处翻转,所以由于弯曲磁力线所导致的磁张力力通过零,并且电阻率与动力学竞争。不稳定性需要电流密度梯度,并
且梯度长度尺度必须比扰动长度尺度k-1小得多。否则,磁张力将稳定该模式。发现最不稳定模式的增长时间大约为$\tau_AS^{3/5}$,相对于全局尺度的电阻层宽度大约为$S^{-2/5}$。
这些依赖于S的特征与Sweet-Parker的比例尺类似,其中3/5和-2/5分别被1/2和-1/2所取代。
Adler、Kulsrud和White(1980)分析了tearing mode的能量学。他们表明,tearing mode降低了磁能,并且驱动能源来自于撕裂层内不稳定的电流梯度。
与Sweet-Parker重联一样,磁能转化为离子流动能量和电子热能量。
随着撕裂的进行,图3中显示的磁岛变宽。一旦岛的宽度超过电阻层的宽度,非线性$J\times B$力就会变得显著。
指数增长被线性增长所取代,其速率与$\eta$成正比(Rutherford 1973)。在这个非常缓慢的增长阶段,最初不稳定的电流分布变平。
当电流分布达到临界稳定度时,该模式饱和。
电阻不稳定性可以被驱动理想不稳定性所改变。电阻屈曲模式,就像理想屈曲一样,由不稳定的电流分布驱动,是一个例子。
它的增长时间大约为$\tau_AS^{1/3}$,比tearing mode快,但在大多数天体物理系统中仍然很慢。就像tearing mode一样,当岛的宽度达到有限值时,
它会从指数增长过渡到代数增长。但电阻屈曲不会饱和,而是形成一个电流层,从而实现快速重联(Waelbroeck,1989年)。
这种两阶段过程是实现快速重联的一种方式,可以被认为是一种驱动重联的形式。
\subsection{Collisionless Reconnection}


\section{磁重联在不同环境中的作用}

\section{磁流体力学方程组}

磁流体力学方程组描述了磁场与等离子体相互作用的基本物理过程。以下是磁流体力学方程组的基本形式:

1. 连续性方程:
$\displaystyle \frac{\partial \rho}{\partial t} + \nabla \cdot (\rho \mathbf{v}) = 0$
其中,$\rho$是等离子体密度,$\mathbf{v}$是等离子体流速。

2. 动量守恒方程:
$\displaystyle \rho \frac{\partial \mathbf{v}}{\partial t} + \rho \mathbf{v} \cdot \nabla \mathbf{v} = -\nabla p + \frac{1}{\mu_0}(\nabla \times \mathbf{B}) \times \mathbf{B} + \rho \mathbf{g}$
其中,$p$是等离子体压力,$\mathbf{B}$是磁场,$\mu_0$是真空磁导率,$\mathbf{g}$是重力加速度。

3. 磁感应方程:
$\displaystyle \frac{\partial \mathbf{B}}{\partial t} = \nabla \times (\mathbf{v} \times \mathbf{B}) - \nabla \times (\eta \nabla \times \mathbf{B})$
其中,$\eta$是等离子体电导率。

4. 能量守恒方程:
$\displaystyle \frac{\partial}{\partial t} (\frac{p}{\rho^\gamma}) + \nabla \cdot (\frac{p}{\rho^\gamma} \mathbf{v}) = \frac{\eta}{\mu_0} (\nabla \times \mathbf{B})^2 + \mathbf{v} \cdot \nabla p + \rho \mathbf{g} \cdot \mathbf{v} + Q$
其中,$\gamma$是等离子体绝热指数,$Q$是能量源项。

\section{MHD equation}

可以使用磁流體動力學 (MHD) 方程從數學上描述磁重聯的原理,這是等離子體物理學的一個分支,研究電離氣體在磁場存在下的行為。
以下是使用一些基本 MHD 方程的簡化解釋:\\
在等離子體中,磁場由磁場矢量 B 描述。磁場線為磁場在每個點處相切的線。 等離子體也由流體速度描述
矢量 V 和等離子體密度 $\rho$。

描述磁場演變的 MHD 方程稱為感應方程:\begin{center}
    $\displaystyle\frac{\partial B}{\partial t} = \nabla\times (V\times B) + \eta \nabla^2 B$
\end{center}

其中,$\displaystyle\frac{\partial B}{\partial t}$ 表示时间偏导数,$\nabla$ 是梯度算子,$\times$ 表示向量叉积,而$\eta$则是磁扩散率,代表等离子体传导磁场的能力。

方程右侧的第一项描述了等离子体流动对磁场的平流作用。第二项表示由于等离子体电阻率引起的磁场扩散。

当来自等离子体不同区域的磁力线相互接近时,磁场可以被压缩和扭曲。这可能会使得磁扩散率变大到足以允许
磁力线断裂并重新连接。在重新连接过程中,磁力线断裂并形成新线路,释放能量并改变了磁场拓扑结构。

可以使用 Sweet-Parker 模型估计重连速率:
\begin{center}
    $\displaystyle vrec \approx v_A(\frac{\delta}{L})^{\frac{1}{2}}$
\end{center}

其中$v_A$是Alfvén velocity,它是磁扰动在等离子体中传播的速度测量值,$\delta$是重连层的厚度,L是系统的特征长度尺度。

总之,磁重联是等离子体中涉及磁场线断裂和合并的基本过程。可以使用MHD方程进行数学描述,并且可以释放大量能量,
这对广泛的天体物理和实验室等离子现象具有重要意义。

\section{磁化等離子體}
磁化等離子體的低頻相對介電常數 $\varepsilon$ 由下式給出
:$\displaystyle \varepsilon = 1 + \frac{c^2\,\mu_0\,\rho}{B^2}$
其中B是磁場強度,$c$是光速,$\mu_0$是真空的Permeability磁導率,質量密度是總和
:$ \displaystyle\rho = \sum_s n_s m_s ,$
所有種類的帶電等離子體粒子(電子以及所有類型的離子)。這裡物種的數量密度為 $n_s$和每個粒子的質量 $m_s$。

電磁波在這種介質中的相速度為
:$\displaystyle v = \frac{c}{\sqrt{\varepsilon}} = \frac{c}{\sqrt{1 + \dfrac{c^2 \mu_0 \rho}{B^2}}}$
對於Alfvén wave的情況
:$\displaystyle v = \frac{v_A}{\sqrt{1 + \dfrac{v_A^2}{c^2}}}$
其中
:$\displaystyle v_A \equiv \frac{B}{\sqrt{\mu_0\,\rho}}$
是Alfvén wave群速度。
(相速度的公式假設等離子體粒子以非相對論速度運動,
參考系中的質量加權粒子速度為零,
並且波平行於磁場矢量傳播。)

如果 $\displaystyle v_A \ll c$,那麼 $v \approx v_A$。
另一方面,當 $\displaystyle v_A \to \infty$ 時,$v \to c$。 即在高場或低密度下,阿爾芬波的群速度接近光速,阿爾芬波成為普通的電磁波。

忽略電子對質量密度的貢獻,
$\rho = n_i\,m_i$,
其中 $n_i$ 是離子數密度,$m_i$ 是每個粒子的平均離子質量,
以便
:$\displaystyle v_A \approx \left(2.18 \times 10^{11}\,\text{cm}\,\text{s}^{-1}\right) \left(\frac{m_i}{m_p}\right)^{-\frac{1}{2}} \left(\frac{n_i}{1~\text{cm}^{-3}}\right)^{-\frac{1}{2} } \left(\frac{B}{1~\text{G}}\right).$













\end{spacing}{}
\bibliographystyle{IEEEtran}
\bibliography{re1}

\end{document}