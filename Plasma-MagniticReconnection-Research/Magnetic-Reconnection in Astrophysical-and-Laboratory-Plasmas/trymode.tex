\documentclass[12pt, a4paper, oneside]{article}
\usepackage{amsmath, amsthm, amssymb, graphicx}
\usepackage[bookmarks=true, colorlinks, citecolor=blue, linkcolor=black]{hyperref}
\usepackage[margin = 25mm]{geometry}
\usepackage{setspace}
\usepackage{listings}
\usepackage{ctex}
\usepackage{cite}
\usepackage{color}


\title{Title}
\date{\today}
\author{Alphabetium}
\begin{document}
\begin{spacing}{2.0}
\tableofcontents
\maketitle


\section{磁重聯简述}

磁重聯是發生在等離子體中的一個過程,最初分開的磁力線聚集在一起並合併,並在此過程中釋放能量。

磁重聯的原理可以用磁力線的概念來解釋。 磁場線就像想像中的線一樣,追踪等離子體中磁場的路徑。 它們有一個方向,可以被認為是從空間中的一點延伸到另一點。

當等離子體不同區域的磁力線聚集在一起時,它們可以重新連接,從而形成新的磁力線。 這個過程釋放能量並改變等離子體中磁場的拓撲結構。 
新的磁場線可以有不同的形狀和方向,這會導致等離子體動力學發生變化。

磁重聯是一個尚未完全理解的複雜過程,但它被認為在許多天體物理現像中起著至關重要的作用,例如太陽耀斑、日冕物質拋射和地球磁層。 
它還與實驗室等離子體、聚變能和空間天氣研究有關。
\section{关于磁重联的理论和模拟研究(重要!!)}
研究人员通过分析和数值模拟对磁重联进行深入研究,以解决天体物理和实验室等离子体中的磁重联现象。此部分涵盖了快速MHD重联、
双流体或无碰撞重联、时间依赖效应、线固定系统、部分电离气体、相对论等离子体和湍流等多个领域。其研究成果对解释天体物理现象,如太阳耀斑、
磁层边界和吸积盘等
\subsection{磁重联的物理机制和基本模型}
\subsection{Sweet-Parker模型}
Sweet-Parker模型描述了在电阻磁流体力学框架下,当重新连接的磁场是反平行(方向相反)且与粘性和可压缩性相关的效应不重要时,
时间无关的磁再连接。初始速度仅为$E\times B$速度,因此:
\begin{center}
    $\displaystyle E_y = v_\text{in} B_\text{in}$
\end{center}
其中$E_y$是垂直于平面的电场,$v_\text{in}$是特征入流速度,而$B_\text{in}$则是特征上游磁场强度。通过忽略位移电流,可以得到低频安培定律。
$\displaystyle\mathbf{J} = \frac{1}{\mu_0}\nabla\times\mathbf{B}$, gives the relation: 
\begin{center}
    $\displaystyle J_y \sim \frac{B_\text{in}}{\mu_0\delta}$
\end{center}
其中$\delta$是电流层的半厚度。该关系使用磁场在约$2\delta$的距离内反转。通过将理想电场与层内阻性电场$\displaystyle\mathbf{E} = \frac{1}{\sigma}\mathbf{J}$(使用欧姆定律)相匹配,我们发现: 
\begin{center}
    $\displaystyle v_\text{in} = \frac{E_y}{B_\text{in}} \sim \frac{1}{\mu_0\sigma\delta} = \frac{\eta}{\delta}$
\end{center}
这里,$\eta$是磁扩散率。当入流密度与出流密度相当时,质量守恒产生以下关系:\begin{center}
    $\displaystyle v_\text{in}L \sim v_\text{out}\delta$
\end{center}
其中$L$是电流层的半长度,$v_\text{out}$是出流速度。上述关系式的左右两侧分别表示进入层和离开层的质量通量。
将上游磁压力与下游动压力相等:\begin{center}
    $\displaystyle \frac{B_\text{in}^2}{2\mu_0} \sim \frac{\rho v_\text{out}^2}{2}$
\end{center}
where $\rho$ is the mass density of the plasma.  Solving for the outflow velocity then gives: 
\begin{center}
    $\displaystyle v_\text{out} \sim \frac{B_\text{in}}{\sqrt{\mu_0\rho}} \equiv v_A$
\end{center}
其中 $v_A$ 是Alfvén wave波or Alfvén velocity。有了上述关系,无量纲重联速率 $R$ 就可以用两种形式来表示,
第一种是使用从欧姆定律中推导出的结果 $(\eta, \delta, v_A)$,第二种是使用质量守恒定律中的 $(\delta, L)$:
\begin{center}
    $\displaystyle R = \frac{v_\text{in}}{v_\text{out}} \sim \frac{\eta}{v_A\delta} \sim \frac{\delta}{L}$
\end{center}
Since the dimensionless Lundquist number $S$ is given by
: 
\begin{center}
    $\displaystyle S \equiv \frac{Lv_A}{\eta}$
\end{center}
$R$的两个不同表达式相乘,然后开方,给出了重新连接速率$R$和Lundquist数$S$之间的简单关系:
\begin{center}
    $\displaystyle R ~ \sim \sqrt{\frac{\eta}{v_A L}} = \frac{1}{S^\frac{1}{2}}$
\end{center}
Sweet-Parker 模型是一種加速機制,在 X 線附近,粒子可以被平行於重新連接的電場自由加速
平面。 如果重聯區延伸距離D,則最大可能的能量增益為 ZeED ,它可能很大,但大多數粒子在達到該能量之前會偏離 X 線。 
這種機制產生的粒子能譜不能很好地符合天體物理光譜,這降低了人們對這種機制的興趣\cite{zweibel2009magnetic}

\subsection{Petschek模型}

Petschek的理论提出,磁重联是通过等离子体中一系列的慢冲击和快冲击发生的,这些冲击将磁能转化为动能和热能。
慢冲击是在磁场线被压缩时形成的,而快冲击是在磁场线被释放并重新连接时形成的。该理论还表明,由于等离子体中湍流的影响,
重新连接过程的发生可能比以前想象的要快得多。

Sweet-Parker模型是比较慢的,因为所有进入重新连接区域的流体都必须通过一个有电阻的通道流出。
Petschek解释如果电阻层很短,大部分进入的流体不通过它流出,而是被静态激波重新定向,重新连接速度会更快。
Petschek的理论被广泛引用来支持快速重新连接,最大重新连接速度是$vA(π/8 \ln S)$,通常是Alfvén velocity 的百分之几,足以解释大多数天体物理现象。
但是,通过数值MHD模拟尝试验证Petschek的理论表明,这种重新连接不会自行发展,除非磁扩散率$\eta$在X点附近增加。
Petschek的模型的一个普遍特点是,与Sweet-Parker模型相比,大部分能量转化为流出的离子动能和热(如果存在激波),相对较少的能量进入电子的电阻加热中。
我們在 2.2 節中簡要討論了 Petschek 的快速 MHD 重聯理論。 該理論基於以下基本見解,即流出區域可能比場線實際重新連接的區域寬得多。 
在最快的 Petschek 模型中,重新連接率幾乎與 S 無關,並且重新連接率接近 vA。 Petschek 重新連接已被證明是不穩定的,除非 $\eta$ 在 X 點附近增加。 
Petschek 解決方案出現問題的第一個跡像是嘗試對其進行數值模擬。 常數 $\eta$ 的模擬發現電阻層的長度是全局的,並且重新連接以緩慢的 Sweet-Parker 
速率進行(Biskamp 1986)。 如果最初設置了 Petschek 配置,它將恢復為 Sweet-Parker 。 然而,如果假設 $\eta$ 在電阻層中增加,
則 Petschek 重聯可以持續 
根據 2.4 節,如果電阻層長度 L 與平均自由程 $\lambda_m f_p$ 相當,則重聯的 MHD 處理無效。 在許多天體物理情況下,
最大速度 Petschek 模型的電流表如此之短,確實如此。 由表 1 的結果可知,在全球尺度 L 為 108 cm、S 為 1010 的日冕環中
,Petschek 理論預測的最小層長約為 30 cm,而 $\lambda_m f_p$ 為 $5 \times 106 (T/106) 2(109/n) cm$ [這似乎與異常電阻率一致 (Uzdensky 2003)]。 
在 L = 1018 cm、S = 1015、n = 1 和 T = 104 的星際氣體中,最小 Petschek 長度約為 80 km,而 $\lambda_m f_p$ 約為 R。 這些例子表明,
即使 Petschek 重聯所需的 $\eta$ 局部增強發生在天體物理環境中,無碰撞效應也可能在達到 Petschek 尺度之前變得重要。
\textcolor{red}{Petschek 體系中 MHD 的分解和對 $\eta$ 的要求的結合意味著 Petschek 重聯的原始形式不太可能是天體物理等離子體重聯的主要模式。 
儘管他關於當流出比電阻層更寬時重聯速度很快的論點是重聯理論中最重要的原則之一,但當今許多重聯研究都集中在非 MHD 效應上。}

\subsection{Spontaneous Reconnection}
Sweet-Parker模型和Petschek模型描述了稳态重联,但没有解释它发生的情况。Furth、Killeen和Rosenbluth(1963)引入了一个新概念,

他们表明磁场可以对小扰动(称为tearing mode)不稳定,并重新连接磁力线。
图3显示了tearing mode的示例。在图的平面方向上有一个强大而几乎均匀的场,
称为引导场。因为平面内的By在$x = 0$处翻转,所以由于弯曲磁力线所导致的磁张力力通过零,并且电阻率与动力学竞争。不稳定性需要电流密度梯度,并
且梯度长度尺度必须比扰动长度尺度k-1小得多。否则,磁张力将稳定该模式。发现最不稳定模式的增长时间大约为$\tau_AS^{3/5}$,
相对于全局尺度的电阻层宽度大约为$S^{-2/5}$。
这些依赖于S的特征与Sweet-Parker的比例尺类似,其中3/5和-2/5分别被1/2和-1/2所取代。
Adler、Kulsrud和White(1980)分析了tearing mode的能量学。他们表明,tearing mode降低了磁能,并且驱动能源来自于撕裂层内不稳定的电流梯度。
与Sweet-Parker重联一样,磁能转化为离子流动能量和电子热能量。
随着撕裂的进行,图3中显示的磁岛变宽。一旦岛的宽度超过电阻层的宽度,非线性$J\times B$力就会变得显著。
指数增长被线性增长所取代,其速率与$\eta$成正比(Rutherford 1973)。在这个非常缓慢的增长阶段,最初不稳定的电流分布变平。
当电流分布达到临界稳定度时,该模式饱和。
电阻不稳定性可以被驱动理想不稳定性所改变。电阻屈曲模式,就像理想屈曲一样,由不稳定的电流分布驱动,是一个例子。
它的增长时间大约为$\tau_AS^{1/3}$,比tearing mode快,但在大多数天体物理系统中仍然很慢。就像tearing mode一样,当岛的宽度达到有限值时,
它会从指数增长过渡到代数增长。但电阻屈曲不会饱和,而是形成一个电流层,从而实现快速重联(Waelbroeck,1989年)。
这种两阶段过程是实现快速重联的一种方式,可以被认为是一种驱动重联的形式。
\subsection{Collisionless Reconnection}
无碰撞重联必须在系统宏观尺度远小于磁场结构的情况下才能发生。此外,在碰撞重联和无碰撞重联中,包括各种类型的不稳定性在内的其他因素也被广泛地研究,
但这些因素不能消除MHD重联时间尺度的瓶颈问题。
在无碰撞重联中观察到了增强的重联速率。通过图12b可以看到,两流体效应在磁重联区域的作用。当断裂的磁力线移向中性线(图12b中的X点)时,离子将失去磁性。
当离子流逐渐改变方向并90°时,电子仍然沿着磁力线移动,直到接近分离线或X点。在第36页中,对磁重联的总结和未解决问题进行了讨论。
其中提到磁重联是等离子物理学的基本过程,在大多数天体物理系统中有着重要作用。文章还研究了重联过程的不同模型以及在各种环境中的实验、
数值模拟和理论。

无碰撞重联是等离子体物理学中的一种现象,即磁场线断裂,然后重新连接,粒子之间没有明显的碰撞。它是许多天体物理和实验室等离子体的一个基本过程,
包括太阳耀斑、地球的磁层和聚变装置。

无碰撞重联研究 "这一术语是指为更好地理解无碰撞重联背后的物理学而进行的理论和实验努力。无碰撞重联的原始理论是在20世纪60年代和70年代发展起来的,
但最近实验和计算技术的进步使研究人员能够重新审视和完善这一理论。

研究无碰撞重联的关键挑战之一是了解磁场线如何在没有碰撞的情况下断开和重新连接。最近的研究表明,湍流和等离子体波可能在促进重联方面发挥了关键作用,
通过加热和加速等离子体中的粒子,并产生能够驱动重联的小尺度磁波动。

无碰撞重联的另一个重要研究领域是研究可能发生的不同类型的重联,包括磁岛的形成和多个重联点之间的互动。
了解这些复杂的过程对于预测和模拟天体物理等离子体的行为以及设计更有效的核聚变装置至关重要。

总的来说,对无碰撞重联的持续研究是等离子体物理学和天体物理学的一个活跃的研究领域,对我们理解宇宙中等离子体的行为具有重要意义。


\subsection{Instabilities and Time-Dependent Effects}
主要探讨了不稳定性和时间演化效应。其中,讨论了微观不稳定性对电子加热、磁场结构以及重联速率的影响,
以及大尺度不稳定性对重联速率的影响和可能导致的爆发现象。在讨论不同尺度中的磁重联过程时提到,虽然在碰撞异质等离子体中的磁重联可能更快,
但在较大的天体等离子体系统中需要纳米级别以下的磁场结构才能实现无碰撞重联,并且需要保证物质流速度与系统尺度的比值保持相对独立。


在等离子体物理学中,不稳定性和时间依赖效应是两个重要的研究领域,它们可以显著影响等离子体的行为。不稳定性指的是等离子体中的小干扰会随着时间的推移而增长和放大,从而导致等离子体行为的重大变化的情况。时间依赖效应是指由于外力或等离子体本身的变化,等离子体的行为随着时间的推移而改变的情况。

等离子体不稳定的一个例子是开尔文-亥姆霍兹不稳定,它发生在两个具有不同速度或密度的等离子体相互作用时。这可能会在两个等离子体之间产生剪切运动,这可能会导致形成旋涡和波浪,随着时间的推移,这些旋涡和波浪可能会增长并变得不稳定。

不稳定的另一个例子是瑞利-泰勒不稳定,它发生在正在加速的等离子体中存在密度梯度的时候。这可能会造成一种情况,即密度较大的物质位于密度较小的物质之上,导致小规模的波和不稳定的形成,这些波和不稳定会随着时间的推移而增长和变得不稳定。

与时间有关的影响可以包括一系列的现象,如等离子体磁场的变化,等离子体温度或密度的变化,或等离子体与外部力量的互动,如电磁波或粒子束。这些影响可以使等离子体的行为随着时间的推移而改变,导致新的不稳定性或改变现有不稳定性的行为。

总的来说,对不稳定性和随时间变化的影响的研究对于理解等离子体在广泛的应用中的行为至关重要,从聚变研究到空间物理学和天体物理学。通过了解这些效应,研究人员可以对等离子体行为开发出更精确的模型和模拟,并设计出更有效的基于等离子体的技术。

\subsection{Instabilities and Turbulent Plasmas}
磁重联是等离子体物理学的一个基本过程,在这个过程中,磁场线断裂,然后重新连接,将储存的磁能释放到等离子体中。
在湍流等离子体中,例如在太阳风和实验室核聚变装置中发现的那些等离子体,磁重联的过程可能特别复杂。

在湍流等离子体中,磁场线可能因等离子体粒子的湍流运动而高度扭曲。这可能导致复杂的电流片的形成,并产生可驱动磁重联的小尺度磁场。

最近对湍流等离子体的研究集中在了解各种物理效应,如等离子体粘度、电阻率和湍流,在驱动磁重联中的作用。研究人员还开发了复杂的计算模型和模拟技术,
以更好地理解这些系统中等离子体的复杂行为。

湍流质体中的磁重联研究的一个重要应用是理解太阳风的行为,这是一股从太阳向外流动并渗透到整个太阳系的带电粒子流。众所周知,太阳风是高度湍流的,
了解这种环境中磁重联的细节对于开发太阳风的精确模型及其与行星和太阳系中其他天体的相互作用至关重要。

总的来说,湍流等离子体中的磁重联研究是等离子体物理学的一个活跃的研究领域,对于理解等离子体在广泛的应用中的行为具有重要的意义,从实验室核聚变装置到太阳风这样的天体物理环境。

Strauss (1988) showed that tearing mode turbulence
can act as a hyper-resistivity, which increases the rate of reconnection on a scale much larger than
the turbulent scale. 

noted that 2D MHD and reduced 3D MHD have
strong conservation laws that constrain turbulence and its interaction with reconnection, but that
the reconnection rate in a full 3D treatment might be quite different.


\subsubsection{Anomalous resistivity}
天体物理和实验等离子体中的磁重联讨论了异常电阻率在增强Sweet-Parker重联速率方面的作用,主要是通过层流霍尔效应。
本文描述了电阻率与等离子体之间的关系,特别是在重联期间。由于高电流密度,电阻性层中的电阻率可以增加,导致增强异常电阻率的不稳定性。此外,
本文介绍了在等离子体中检测到的各种类型波,在重连过程中包括较低杂交漂移不稳定性和电磁较低杂交波。虽然涨落水平与重连速率之间的相关性仍难以建立,
但检测到的鸣叫声已经显示出与磁重联实验中重新连接速率呈正相关关系

并且是当前激烈争论的话题。
\subsubsection{Reconnection in Line-Tied Systems}
不稳定性也可以产生湍流,加热和加速粒子。能量化方法的详细讨论超出了本文的范围,但我们提到了一些贡献。
Ambrosiano等人在湍流MHD重连模型中使用测试粒子方法计算了加速度(1988年)。最近,de Gouveia dal Pino&Lazarian(2005)讨论了电子的一阶费米加速,
并在Drake等人对无碰撞重连模型进行研究时进行了研究(2006)。在锯齿崩溃期间,在MST中观察到伴随着强离子加热的电磁波动增强;
Tangri、Terry和Fiksel(2008)考虑了这种情况下各种离子加热机制,但没有找到完全解释数据的模型。

由于相邻电流丝之间的相互吸引力(Finn&Kaw 1977),图3所示由于磁撕裂而形成的磁岛链可能不稳定并合并。重新连接层中的外流可能对Kelvin-Helmholtz模式不稳定,
从而使它们变得湍流(Chiueh&Zweibel 1987)。Sweet-Parker重新连接层中可能会发展出撕裂模式(Bulanov,Sakai&Syrovatskii 1979;Biskamp 1993;
Malyshkin,Linde和Kulsrud 2005)。Daughton、Scudder和Karimabadi(2006)在具有开放边界条件的无碰撞重连的动力学模拟中表明,电子扩散区随时间逐渐变长,
并最终不稳定地形成分离的等离子体小团块被喷出系统,导致重新连接速率不稳定。Loureiro、Schekochihin和Cowley(2007)
也讨论了MHD重连期间等离子体小团块的形成。

尽管这些波动对于确定重新连接系统如何演化非常重要,但它们并不能消除由长时间MHD重新连接时间尺度所施加的瓶颈。这个问题必须通过其他过程来解决,
\subsection{Reconnection in Line-Tied Systems}
磁重联是等离子体物理学的一个基本过程,在这个过程中,磁场线断裂,然后重新连接,将储存的磁能释放到等离子体中。
在线束系统中,例如锚定在固体物体或边界上的磁场,磁重联的过程可能特别复杂。

在线束系统中,当磁场线断裂,然后在锚定到固体边界的区域重新连接时,就会发生磁重联。这可能发生在各种等离子体环境中,如日冕、地球的磁层和实验室核聚变装置。

由于磁场线受到边界存在的强烈影响,线状系统中的磁重联过程在研究和理解上可能特别具有挑战性。这可能会导致电流片和其他复杂结构的形成,
从而大大影响等离子体的行为。

最近对线状系统的研究集中在理解各种物理效应的作用上,如等离子体电阻率、等离子体粘度和等离子体不稳定性,以驱动磁重联。
研究人员还开发了复杂的计算模型和模拟技术,以更好地理解这些系统中等离子体的复杂行为。

总的来说,线束系统中的重联研究是等离子体物理学的一个活跃的研究领域,对理解等离子体在广泛的应用中的行为有着重要的意义,从聚变研究到空间物理学和天体物理学。


\subsection{Reconnection in Weakly Ionized Gases}
磁重联是等离子体物理学的一个基本过程,在这个过程中,磁场线断裂,然后重新连接,将储存的磁能释放到等离子体中。在弱电离气体中,
例如在一些实验室等离子体和行星的高层大气中发现的那些气体,磁重联的过程可能特别复杂。

在弱电离气体中,等离子体的行为受到中性粒子存在的强烈影响,这可能会大大影响等离子体的导电性和其他物理特性。这可能使磁重联的过程更难研究和理解。

最近对弱电离气体的研究集中在理解各种物理效应的作用上,如双极扩散、中性阻力和离子-中性碰撞,以驱动磁重联。非极性扩散是指带电粒子相对于中性粒子的运动,
而中性阻力是指等离子体与周围气体中中性粒子的相互作用。离子-中性碰撞也可以在影响等离子体的行为方面发挥重要作用。

研究人员已经开发了复杂的实验技术和计算模型,以更好地理解弱电离气体中等离子体的复杂行为。这些努力揭示了一些有趣的现象,如薄电流片的形成和高能粒子的产生,
这些现象在这些系统的磁重联期间可能发生。

总的来说,弱电离气体中的磁重联研究是等离子体物理学的一个活跃的研究领域,对于理解等离子体在从实验室核聚变装置到行星大气等广泛的应用中的行为具有重要意义。
\subsection{Reconnection in Relativistic Plasmas}
磁重联是等离子体物理学的一个基本过程,在这个过程中,磁场线断裂,然后重新连接,将储存的磁能释放到等离子体中。在相对论等离子体中,
例如在天体物理环境中发现的那些等离子体,磁重联的过程可能特别复杂,对这些等离子体的行为有重大影响。

在相对论质体中,等离子体粒子可以接近光速,这可以导致一些独特的现象。这些现象包括产生高能粒子,产生强烈的电磁场,以及形成冲击波和其他结构。

最近对相对论等离子体的研究集中在理解各种物理效应的作用上,如强磁场的存在、狭义相对论的影响以及粒子加速机制的影响,以驱动磁重联。
研究人员还开发了复杂的计算模型和模拟技术,以更好地理解这些系统中等离子体的复杂行为。

相对论等离子体中的磁重联研究的一个重要应用是理解等离子体喷流的行为,这些喷流在一些天体物理环境中被观察到,如活动星系核和伽玛射线暴。
这些喷流被认为是由磁重联过程驱动的,了解这些过程的细节对于开发这些现象的精确模型至关重要。

总的来说,相对论等离子体中的磁重联研究是等离子体物理学的一个活跃的研究领域,对理解广泛的天体物理环境中的等离子体行为具有重要意义。






\section{Summary}
磁重联是等离子体物理学中的基本过程。由于大多数天体物理系统的兰德奎斯特数S排除了磁通耗散等电阻行为,因此它也是天体物理学中的关键过程。
它从系统中提取磁能以驱动实验室锯齿形崩溃和天体爆发,并为磁自组织过程(包括发电机和泰勒弛豫)提供必要的磁拓扑变化。实验室实验、太阳耀斑详细观测、
空间等离子体原位测量、数值模拟和理论都为研究重联提供了良好平台。

我们简要描述了几种重联模型。Sweet-Parker理论是一种MHD理论,其特点是长而薄的电流层、缓慢的流入和Alfvenic流出,由于所有进入电阻层的流体
都必须通过一个窄通道以vA被排出,所以速度较慢。Petschek's MHD 理论通过使电流片更短并通过冲击波转移液体来解决连续性问题。但只有当等离子体电阻率$\eta$
向湮灭区域增加时,Petschek重联才能实现。虽然在无碰撞重联中可以达到类似Petschek的流动,但基本物理学是完全不同的,离子动力学不允许冲击结构。
Sweet-Parker理论中出现的连续性问题的其他解决方法包括相对论提升(第5.6节)、等离子体复合(第5.5节)和磁场线湍流扩散(第5.7节)。

长期以来,在为什么无碰撞等离子体中发生重联如此之快方面已经考虑了两种流体物理学。Hall MHD效应在数值模拟、实验室和空间等离子体中得到验证。
在碰撞区域,Sweet-Parker模型已通过数值模拟和实验室实验证明。发现随着电子平均自由程与比例尺长度之比增加,重连接速率迅速增加。
这个结果归因于重新连接层内大Hall电场除X点附近外强烈耗散机制所起作用。

本文继续探讨一个主题:为了使任何快速重联理论在天体物理学中运行,必须有产生小尺度结构的机制——无论是为了解耦电子和离子,
在无碰撞重联中是必要的(第5.2节),维持异常电阻所需的高电流密度(第5.3.1节)还是避免某些理论预测的重新连接速率与全局长度尺度不利的比例关系。
虽然我们只简短地概述了这些机制(第6.1节),但不应忘记它们的需要。

有几个问题需要进一步研究。我们需要更全面地了解电子耗散区的结构和动力学,即磁场线断裂的区域。GEM挑战项目(Birn等人,2001)的共识是重新连接速率由
广义欧姆定律中的霍尔项控制。然而,这个术语并不提供能量耗散。在最近使用粒子在单元格(PIC)代码进行的研究中,我们已经了解到,在中性片区内发生能量耗散
将发生在一个小区域内,导致从磁场到粒子动能转换的速率要小得多。这个速率太小了,无法解释RFP等离子体松弛事件、球形融合实验或太阳耀斑演化期间观察到的重
新连接过程中观察到的粒子加热。
目前还没有清晰明确地理论说明如何将磁能转换为等离子体动能。我们还不知道波动起什么作用、它们是如何被激发以及它们通过影响能量转换过程来确定重新连接速率
。调查异常粒子加速与重连速率之间关系非常重要。这也对预测重联时观测特征非常重要,包括离子速度剖面和电子分布函数的光谱特征。这些是诊断天体物理环境中重
新连接的基石。
我们需要更好地了解重新连接如何与全球系统耦合。虽然通过几乎同时和冲动地触发多个重连站点可以缓解许多涉及太阳耀斑能量学的困难,但我们还不知道这是如何发
生的。是否有任何一般标准或原因导致磁能被储存长时间,然后突然释放,将等离子体驱动到全局松弛状态?本地重联速率与全局储存能量积累之间的关系是否至关重要?
快速无碰撞重联在宏观尺寸远大于平均自由程的系统中是否可行,这在天体物理学中经常出现?
等离子体的磁性自组织既受到重新连接区域内局部等离子体动力学影响和3D全球拓扑边界条件影响和确定。实验室等离子体已经发现由外部条件引起的大型MHD不稳定性
通常会产生一个电流层,在其中进行磁再连接并可以确定其速率。在这个领域,通过实验研究实验室等离子体的磁性自组织的关键特征,并比较快速发展的先进数值模拟结果
,可以取得重大进展。
我们需要更详细地了解重新连接在通常适用于天体物理学但不适用于实验室等离子体(相对论等离子体、弱电离气体和不包含强流动但包含有序和湍流流动的系统)中
的情况。理论家和实验家之间的合作可以回答这些问题。改善对磁再连接物理学的理解应该为天体物理学家提供工具来开发耀斑、天文发电机、$\gamma$射线暴以及吸积盘演
化的理论,并解释它们的观测特征。鉴于目前的进展速度,我们对所有这些问题的进一步发展持乐观态度。




\section{磁流体力学方程组}

磁流体力学方程组描述了磁场与等离子体相互作用的基本物理过程。以下是磁流体力学方程组的基本形式:

1. 连续性方程:
$\displaystyle \frac{\partial \rho}{\partial t} + \nabla \cdot (\rho \mathbf{v}) = 0$
其中,$\rho$是等离子体密度,$\mathbf{v}$是等离子体流速。

2. 动量守恒方程:
$\displaystyle \rho \frac{\partial \mathbf{v}}{\partial t} + \rho \mathbf{v} \cdot \nabla \mathbf{v} = -\nabla p + \frac{1}{\mu_0}(\nabla \times \mathbf{B}) \times \mathbf{B} + \rho \mathbf{g}$
其中,$p$是等离子体压力,$\mathbf{B}$是磁场,$\mu_0$是真空磁导率,$\mathbf{g}$是重力加速度。

3. 磁感应方程:
$\displaystyle \frac{\partial \mathbf{B}}{\partial t} = \nabla \times (\mathbf{v} \times \mathbf{B}) - \nabla \times (\eta \nabla \times \mathbf{B})$
其中,$\eta$是等离子体电导率。

4. 能量守恒方程:
$\displaystyle \frac{\partial}{\partial t} (\frac{p}{\rho^\gamma}) + \nabla \cdot (\frac{p}{\rho^\gamma} \mathbf{v}) = \frac{\eta}{\mu_0} (\nabla \times \mathbf{B})^2 + \mathbf{v} \cdot \nabla p + \rho \mathbf{g} \cdot \mathbf{v} + Q$
其中,$\gamma$是等离子体绝热指数,$Q$是能量源项。

\section{MHD equation}

可以使用磁流體動力學 (MHD) 方程從數學上描述磁重聯的原理,這是等離子體物理學的一個分支,研究電離氣體在磁場存在下的行為。
以下是使用一些基本 MHD 方程的簡化解釋:\\
在等離子體中,磁場由磁場矢量 B 描述。磁場線為磁場在每個點處相切的線。 等離子體也由流體速度描述
矢量 V 和等離子體密度 $\rho$。

描述磁場演變的 MHD 方程稱為感應方程:\begin{center}
    $\displaystyle\frac{\partial B}{\partial t} = \nabla\times (V\times B) + \eta \nabla^2 B$
\end{center}

其中,$\displaystyle\frac{\partial B}{\partial t}$ 表示时间偏导数,$\nabla$ 是梯度算子,$\times$ 表示向量叉积,而$\eta$则是磁扩散率,代表等离子体传导磁场的能力。

方程右侧的第一项描述了等离子体流动对磁场的平流作用。第二项表示由于等离子体电阻率引起的磁场扩散。

当来自等离子体不同区域的磁力线相互接近时,磁场可以被压缩和扭曲。这可能会使得磁扩散率变大到足以允许
磁力线断裂并重新连接。在重新连接过程中,磁力线断裂并形成新线路,释放能量并改变了磁场拓扑结构。

可以使用 Sweet-Parker 模型估计重连速率:
\begin{center}
    $\displaystyle vrec \approx v_A(\frac{\delta}{L})^{\frac{1}{2}}$
\end{center}

其中$v_A$是Alfvén velocity,它是磁扰动在等离子体中传播的速度测量值,$\delta$是重连层的厚度,L是系统的特征长度尺度。

总之,磁重联是等离子体中涉及磁场线断裂和合并的基本过程。可以使用MHD方程进行数学描述,并且可以释放大量能量,
这对广泛的天体物理和实验室等离子现象具有重要意义。

\section{磁化等離子體}
磁化等離子體的低頻相對介電常數 $\varepsilon$ 由下式給出
:$\displaystyle \varepsilon = 1 + \frac{c^2\,\mu_0\,\rho}{B^2}$
其中B是磁場強度,$c$是光速,$\mu_0$是真空的Permeability磁導率,質量密度是總和
:$ \displaystyle\rho = \sum_s n_s m_s ,$
所有種類的帶電等離子體粒子(電子以及所有類型的離子)。這裡物種的數量密度為 $n_s$和每個粒子的質量 $m_s$。

電磁波在這種介質中的相速度為
:$\displaystyle v = \frac{c}{\sqrt{\varepsilon}} = \frac{c}{\sqrt{1 + \dfrac{c^2 \mu_0 \rho}{B^2}}}$
對於Alfvén wave的情況
:$\displaystyle v = \frac{v_A}{\sqrt{1 + \dfrac{v_A^2}{c^2}}}$
其中
:$\displaystyle v_A \equiv \frac{B}{\sqrt{\mu_0\,\rho}}$
是Alfvén wave群速度。
(相速度的公式假設等離子體粒子以非相對論速度運動,
參考系中的質量加權粒子速度為零,
並且波平行於磁場矢量傳播。)

如果 $\displaystyle v_A \ll c$,那麼 $v \approx v_A$。
另一方面,當 $\displaystyle v_A \to \infty$ 時,$v \to c$。 即在高場或低密度下,阿爾芬波的群速度接近光速,阿爾芬波成為普通的電磁波。

忽略電子對質量密度的貢獻,
$\rho = n_i\,m_i$,
其中 $n_i$ 是離子數密度,$m_i$ 是每個粒子的平均離子質量,
以便
:$\displaystyle v_A \approx \left(2.18 \times 10^{11}\,\text{cm}\,\text{s}^{-1}\right) \left(\frac{m_i}{m_p}\right)^{-\frac{1}{2}} \left(\frac{n_i}{1~\text{cm}^{-3}}\right)^{-\frac{1}{2} } \left(\frac{B}{1~\text{G}}\right).$













\end{spacing}{}
\bibliographystyle{IEEEtran}
\bibliography{re1}

\end{document}