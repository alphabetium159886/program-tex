\documentclass[12pt, a4paper, oneside]{ctexart}
\usepackage{amsmath, amsthm, amssymb, graphicx}
\usepackage[bookmarks=true, colorlinks, citecolor=blue, linkcolor=black]{hyperref}
\usepackage[margin = 25mm]{geometry}
\usepackage{setspace}
\usepackage{listings}
\usepackage{float}


\title{期末题目}
\date{\today}
\author{202011010101物理2001孙陶庵}
\begin{document}
\begin{spacing}{2.0}
\tableofcontents
\maketitle

\section{Dynamos}
大多数(如果不是全部)天体物理环境中都存在磁场。 所以显而易见的问题是“为什么?”。 
两个答案浮现在脑海中:要么宇宙诞生时就带有“原始”磁场,但这些磁场从那时起就一直在衰减,要么就是有什么东西在产生它们。 
事实上,原始场有可能仍然潜伏在周围,但有充分的证据表明场是可以产生的。 在
实验室中,简单的电路和电流会根据培尔定律(公式 2.2)产生磁场。 
然而,在等离子体中,场更经常通过感应方程 (4.13) 中体现的电动效应感应。 这构成了研究磁场产生的基础,通常称为“发电机问题”。
\begin{center}
    $\displaystyle\frac{\partial\mathbf B}{\partial t}=\nabla\times(\mathbf V\times\mathbf B)+\eta\nabla^2\mathbf B$
\end{center}
其中我们引入了磁扩散率 η ≡ 1/(µ0σ) 来强调这一项的作用。右侧的第二项表示磁扩散,它趋向于平滑和消散磁场。 该项类似于热在传导介质中的扩散。
第一项$\nabla\times(\mathbf V\times\mathbf B)$表示流体速度对磁力线的拉伸和扭曲。 它描述了导电流体的运动如何放大和维持磁场。 
该项负责发电机作用,其中流体运动通过磁感应过程产生并维持磁场。 正如我们在 4.2 节中看到的,该方程式的最后一项表示磁扩散,而中间项
项等离子体整体流动的场对流。 然后由于扩散引起的磁场衰减(这也导致场的耗散)产生特征衰减时间 τ 从 (4.69) 中找到作为
\begin{center}
    $\displaystyle\frac{B}{\tau}\sim\eta\frac{B}{L^2}$
\end{center}
使用标准(碰撞)值来评估 τ 给出了很长但不是无限的时间尺度。 在地核,$\tau \thicksim 3\times 10^5$年; 在太阳中,$\tau \thicksim 10^11$年,仅略长于太阳的年龄。 
这个简单的计算表明地球需要一些东西来补充它的磁场,但也许太阳不需要。 然而,我们知道太阳磁场会在短时间内(11 年)发生反转,
地球也会发生反转,尽管没有那么频繁也没有规律。 因此,一些活跃的过程正在产生这些领域。 发电机确实存在。 
唯一的问题是从我们的数学和等离子体物理理解来解释它们。
\section{The Dynamo Problem}
\textcolor{red}{完整的发电机问题包括使用 MHD 方程来显示流体运动 V 和磁场 B 如何自洽和自保持。 运动通过感应磁场
上面给出的感应方程,而场又通过 MHD 动量方程 (4.2) 影响运动。 这个问题是高度非线性的,即使在数值上也很难处理。 即便是
归纳方程 (4.69) 本身因非线性对流项而变得复杂。
对这个完整问题的妥协是考虑运动发电机问题,其中规定了流动并探索了磁场的后果。 这使得 (4.69) 在未知场行为中呈线性,
并且能够取得一些进展,这有望阐明完整的发电机问题。}
已知的天体物理场在结构上往往是偶极的,并且大致与对称/旋转轴对齐,因此寻求具有这些特性的解决方案是很自然的。 
不幸的是,由于考林的结果,称为考林定理,介入了。 考林表明,对于具有层流场和流动的发电机,不可能有稳定的轴对称解!
由于 Cowling 定理适用于精确场,因此可能存在某些平均场和流动轴对称的解。 这催生了大量关于平均场电动力学主题的文献,其中插入并研究了湍流引起的波动特性。 
我们将在本节中回到这个主题


\section{Qualitative Dynamo Behaviour}
虽然发电机行为的数学很快变得复杂,但定性过程
可以相当简单地说明并由以下操作序列组成,如图所示
\\
1.拉伸场线以延长它们。 这个过程确实在现场工作,将流动能量转移到磁能密度。\\
2.扭曲拉伸场,将拉伸场的方向更改为原始场的方向。进一步的扭曲或折叠会放大原来的场域。 这是关键的一步。\\
3.将场扩散到上述步骤中施加的小尺度上以返回系统朝着它的起点\\


\section{Mean Field Kinematic Dynamos}
我们在这里限制自己可能是发电机问题的最简单的数学公式。这是一种纯运动学方法,其中规定了速度场,我们寻求由此产生的
磁场通过观察感应方程。 我们通过将所有数量分解为它们来进行平均值(下标“0”)及其波动(下标“1”)。 因此我们写

\begin{center}
    $\displaystyle\begin{array}{rcl}\mathbf{B}&=&\mathbf{B}_0+\mathbf{B}_1\\ 
        \mathbf{V}&=&\mathbf{V}_0+\mathbf{V}_1\end{array}$
\end{center}
其中$B_0\equiv <B>$,平均在空间、时间或系综(我们不需要区分这些)上,因此$<B_1> = 0$,这让人想起我们在第 4.6.1 节中执行的线性化
除了这里没有假设波动量$B_1, V_1$非常小,代入归纳方程得到

\begin{center}
    $\displaystyle\frac{\partial}{\partial t}\left(\mathbf{B}_0+\mathbf{B}_1\right)=\nabla\times\left[\left(\mathbf{V}_0+\mathbf{V}_1\right)\times\left(\mathbf{B}_0+\mathbf{B}_1\right)\right]+\eta\nabla^2\left(\mathbf{B}_0+\mathbf{B}_1\right)$
\end{center}

取该方程的平均值得出以下控制平均磁场的方程
\begin{center}
    $\displaystyle\frac{\partial\mathbf{B}_0}{\partial t}=\nabla\times(\mathbf{V}_0\times\mathbf{B}_0)+\nabla\times(<\mathbf{V}_1\times\mathbf{B}_1>)+\eta\nabla^2\mathbf{B}_0$
\end{center}

因此我们可以看到,V 和 B 的波动会产生一个额外的电动势 $E ≡ < V1 ×B1 >$,这会对感应磁场产生贡献。 
E 的细节取决于波动的性质,通常假定为湍流。 特别地,如果平均场是轴对称的,波动不能是为了避免考林定理的约束。 
动量方程的说明(以确定 V1)以及 (4.73) 和 (4.74) 之间的差异(以确定 B1)超出了本文的范围。 然而,相当一般的论证得出的结论是 E 的形式为

\begin{center}
    $\displaystyle\frac{\partial\mathbf{B}_0}{\partial t}=\nabla\times\left(\mathbf{V}_0\times\mathbf{B}_0\right)+\nabla\times\left(\alpha\mathbf{B}_0\right)+\left(\mathbf{\eta}+\mathbf{\eta}_T\right)\nabla^2\mathbf{B}_0$
\end{center}
其中我们引入了由 E 中的 ββ 项产生的湍流扩散率 ηT。鉴于基于经典扩散率的衰减时间通常很长,如果要达到场生成和耗散之间的平衡,
介质中湍流的这种作用是必不可少的 . 方程式 4.76 构成了许多发电机作用研究的起点。 就图 4.9 中描绘的发电机步骤而言,右侧的第一项体现了平均流对场的拉伸,
而第二项包含由于湍流中的螺旋性而导致的“$\alpha$ 效应”扭曲。
\section{$\alpha−\omega$ Solar Dynamo}

太阳能发电机代表了对任何发电机理论的关键挑战。 最简单的数学方法从 (4.76) 开始,并做出以下进一步的简化/假设:
\\
1.平均场 B0 和 V0 是轴对称的,即它们在球极坐标表示中不依赖于 φ。 特别地,我们取 $B_0 = B_0(r, \theta)$\\
2.平均速度场是一种径向相关的差分旋转流体$V_0 = \omega(r, \theta)r \sin(\theta)\hat{\phi}$

\begin{center}
    $\displaystyle\nabla\times(\mathbf{V}_0\times\mathbf{B}_0)=\hat{\phi}\sin\theta\left[r B_r\frac{\partial\omega}{\partial r}+B_0\frac{\partial\omega}{\partial\theta}\right]$
\end{center}

\begin{center}
    $\displaystyle\frac{\partial\mathbf{B}_0}{\partial t}=\hat{\phi}\sin\theta\left[rB_r\frac{\partial\mathbf{\omega}}{\partial r}+B_0\frac{\partial\mathbf{\omega}}{\partial\mathbf{\theta}}\right]+\nabla\times\left(\alpha\mathbf{B}_0\right)+\left(\mathfrak{\eta}+\mathfrak{\eta}_T\right)\nabla^2\mathbf{B}_0$
\end{center}

右侧的第一项体现了$\omega$效应,它结束初始极向场以生成环形$ (\phi) $极向场。 第二项描述了$\alpha$效应,其中湍流运动导致场的扭曲。 
在太阳系的情况下,净扭曲(与相等和抵消扭曲相反)是由分层太阳大气中上升通量的科里奥利效应实现的,该效应与下降通量管上的上升通量不对称。 这些影响如图所示


\section{Astrophysical Dynamos}

我们已经探索了太阳能发电机的许多方面。 11 年的逆转对发电机理论构成了重大挑战。 此外,在该周期内,
太阳黑子等强磁场集中出现在周期早期的中纬度地区,但随着周期的进行,活动区越来越多地出现在低纬度地区。 这产生了图 4.11 中所示的“蝴蝶”模式。 
太阳磁场的另一个特征是新兴通量区域的极性。 这种极性遵循规则模式,主要极性在相反的半球相反。 如图 4.12 所示,这种感觉随着每个太阳周期的逆转而逆转。 
这些事实被称为黑尔定律,与图 4.10 左侧部分描绘的差动旋转产生的环形场定性一致,然后由于磁浮力而上升

涉及圆盘的天体物理系统,例如吸积现象,也是发电机作用的场所。 在这些情况下,运动学不稳定性会引起湍流流体运动,进而通过发电机作用产生场。 
然而,这些场是纯粹的湍流场,没有大尺度的平均场。 尽管如此,湍流扩散率(对于物质和场)在角动量传输和吸积率方面对这些系统的演化起着至关重要的作用。


\section{Antidynamo theorem or Cowling's theorem}

因为没有能放大磁场的简单速度场。对实际问题,这意味着考察的系统对称性不能太高。

\textcolor{red}{在标准不可压缩MHD的框架下,轴对称磁场不能由轴对称速度场维持。}
不可能有简单的解给出发电机效应。这些定理例如,“二维流不能维持磁场”,“无关空间坐标的磁场不能维持”。


\section{放大作用}
MHD Dynamo理论中的放大机制主要涉及磁感应方程中的两个项:扩散项和涡旋项。

首先,扩散项代表了磁场的扩散或耗散过程,它通过磁扩散系数来描述。磁扩散会导致磁场在空间中逐渐衰减和消失,这是一个抑制磁场放大的过程。

然而,涡旋项是MHD Dynamo理论中的关键机制,它描述了磁场与流体速度之间的相互作用。
当流体运动产生涡旋时,磁场可以通过速度场的涡旋项进行放大。具体来说,当流体速度与磁场的叉乘产生涡旋时,
涡旋项会将磁场的一部分转移到新的区域,并且随着时间的推移,磁场可以在整个系统中得到放大。

综合考虑扩散项和涡旋项,MHD Dynamo理论描述了磁场如何通过流体运动和相互作用来放大。在一定的条件下,
流体运动和涡旋可以生成足够的磁场强度,使得磁场能够在系统中得到持续放大,形成稳定和持久的磁场结构。


\section{磁重联}
|作為電阻等離子體中的一個基本過程,磁重聯 [Vasyliunas, 1975; Biskamp, 1993] 長期以來一直被認為是決定太陽區域 [例如 Parker, 1979]、
磁層亞暴 [例如 Akasofu, 1972] 和實驗室等離子體弛豫現象 [例如 Taylor, 1974] 動力學的關鍵機制.儘管這是一個局部過程,
但它通常會通過磁力線的切割和重新連接而引起宏觀磁場拓撲結構的根本變化。因此,必須存在全局螺旋性和局部重聯事件之間的內在關係。
發電機效應也是導電流體或等離子體研究的另一個焦點,試圖解釋觀測到的太陽和行星磁場。特別是,通過湍流或眾所周知的效應 [Parker,1955] 
沿平均場產生電動勢 (EMF),是放大大規模磁場的重要過程 [例如,Proctor 和 Gilbert, 1994]。這些發電機效應驅動並聯電流扭曲磁力線,從
而在大尺度上產生磁螺旋性。因此,螺旋度也必然與發電機效應密切相關。

这段文字总结如下:

磁重联是一个基本过程,被认为是决定太阳区域、磁层亚暴和实验室等离子体弛豫现象动力学的关键机制。
它通过磁力线的切割和重新连接引起宏观磁场拓扑结构的变化。全局螺旋性和局部重联事件之间存在内在关系。

发电机效应是导电流体或等离子体研究的另一个焦点,试图解释观测到的太阳和行星磁场。通过湍流或已知效应,
沿平均场产生电动势(EMF)是放大大尺度磁场的重要过程。这些发电机效应驱动并扭曲磁力线的并联电流,从而在大尺度上产生磁螺旋性。
因此,螺旋度与发电机效应密切相关。

根据上述文字,磁重联和发电机效应在这里存在一种内在的关系。磁重联是一个基本过程,通过切割和重新连接磁力线,
引起宏观磁场拓扑结构的变化。发电机效应是指通过湍流或已知效应,在导电流体或等离子体中产生电动势(EMF),从而放大大尺度磁场。

具体而言,发电机效应驱动并扭曲磁力线的并联电流,进而产生磁螺旋性。这表明,发电机效应在一定程度上涉及磁重联过程。
通过湍流或已知效应产生的电动势使得磁力线重新连接并形成螺旋结构,从而在大尺度上增强磁场。

因此,磁重联和发电机效应相互作用,共同参与了磁场的生成和演化过程。磁重联提供了磁场拓扑变化的基础,而发电机效应驱动了磁场的增强和扭曲。
这种相互作用在解释天体物理现象和实验室等离子体行为方面起着重要作用。




\section{但是这样的话omega和alpha效应的区别是什么}
区别:

引起效应的原因不同:$\omega$效应是由于星体的自转引起的,而$\alpha$效应是由湍流运动引起的。
物理机制不同:$\omega$效应是由于自转引起的初始极向磁场的涡旋效应,生成环向磁场;$\alpha$效应是由湍流运动扭曲磁场线,产生额外的磁场扭曲效应。
对磁场的影响不同:$\omega$效应主要影响磁场的转向和演化,而$\alpha$效应主要影响磁场的形状和扭曲程度。
\section{氢气柱密度}

氢气柱密度是指单位面积上通过的氢气质量或氢原子数量。它通常用于描述天体物理学中的天体或星际介质中的氢气分布情况。

氢气柱密度常用单位是每平方厘米或每平方秒角(square centimeter or square arcsecond)。在天文学中,观测天体的亮度可以通过测量通过视线上的单位面积的氢气柱密度来推断。

氢气柱密度的具体值取决于所研究的天体或天体区域的特性和性质。例如,在星际介质中,氢气柱密度可以在几个数量级上变化,从较低的值(约为1个原子/平方厘米)到较高的值(约为$10^23$个原子/平方厘米)。

请注意,氢气柱密度是一个表示氢气分布的局部参数,并且它通常与其他天体物理学参数(如气体温度、密度分布、速度场等)相互关联。在具体的天体或星际介质研究中,通过多种观测手段(如射电波段、光学波段等)和物理模型,可以推断出氢气柱密度的估计值。






\end{spacing}{}

\end{document}