\documentclass[12pt, a4paper, oneside]{article}
\usepackage{amsmath, amsthm, amssymb, graphicx}
\usepackage[bookmarks=true, colorlinks, citecolor=blue, linkcolor=black]{hyperref}
\usepackage[margin = 25mm]{geometry}
\usepackage{setspace}
\usepackage{listings}
\usepackage{cite}
\usepackage{ctex}
\usepackage{xcolor}

\definecolor{codegreen}{rgb}{0,0.6,0}
\definecolor{codegray}{rgb}{0.5,0.5,0.5}
\definecolor{codepurple}{rgb}{0.58,0,0.82}
\definecolor{backcolour}{rgb}{0.95,0.95,0.92}

\lstdefinestyle{mystyle}{
    backgroundcolor=\color{backcolour},   
    commentstyle=\color{codegreen},
    keywordstyle=\color{magenta},
    numberstyle=\tiny\color{codegray},
    stringstyle=\color{codepurple},
    basicstyle=\ttfamily\footnotesize,
    breakatwhitespace=false,         
    breaklines=true,                 
    captionpos=b,                    
    keepspaces=true,                 
    numbers=left,                    
    numbersep=5pt,                  
    showspaces=false,                
    showstringspaces=false,
    showtabs=false,                  
    tabsize=2
}

\lstset{style=mystyle}


\title{Title}
\date{\today}
\author{Alphabetium}
\begin{document}
\begin{spacing}{2.0}
\maketitle


\section{用定义函数的形式写出jacobi方法}

\begin{lstlisting}[language=Python, caption=0.1s]
    import numpy as np

    def jacobi(A, b, x0, max_iter=1000, eps=1e-8):
        """
        Jacobi method for solving linear equations Ax = b.
        A: coefficient matrix
        b: constant vector
        x0: initial approximation
        max_iter: maximum number of iterations
        eps: tolerance for stopping criterion
        """
        n = A.shape[0]  # dimension of the problem
        x = x0.copy()  # make a copy of the initial approximation
        for k in range(max_iter):
            x1 = np.zeros_like(x)
            for i in range(n):
                x1[i] = (b[i] - np.dot(A[i, :i], x[:i]) - np.dot(A[i, i+1:], x[i+1:])) / A[i, i]
            if np.linalg.norm(x1 - x) < eps:
                break
            x = x1
        return x
    
\end{lstlisting}
Q:详细解释np.dot(A[i, :i], x[:i])\\
A:\\
np.dot(A[i, :i], x[:i]) 是计算向量内积的 numpy 函数。在这里,它用于计算 $A$ 矩阵第 $i$ 行的前 $i$ 个元素和当前迭代向量 $x$ 的前 $i$ 个分量的内积。

具体来说,A[i, :i] 是 $A$ 矩阵第 $i$ 行的前 $i$ 个元素,x[:i] 是当前迭代向量 $x$ 的前 $i$ 个分量,因此 np.dot(A[i, :i], x[:i]) 就是这两个向量的内积。它的计算公式为:

$\sum_{j=0}^{i-1} A_{ij} x_j$

其中,$A_{ij}$ 是矩阵 $A$ 的第 $i$ 行第 $j$ 列的元素,$x_j$ 是向量 $x$ 的第 $j$ 个分量。\\

Q:详细解释np.dot(A[i, i+1:], x[i+1:])\\
A:\\
np.dot(A[i, i+1:], x[i+1:]) 同样是计算向量内积的 numpy 函数。在这里,它用于计算 $A$ 矩阵第 $i$ 行的后 $n-i-1$ 个元素和当前迭代向量 $x$ 的后 $n-i-1$ 个分量的内积。

具体来说,A[i, i+1:] 是 $A$ 矩阵第 $i$ 行的后 $n-i-1$ 个元素,x[i+1:] 是当前迭代向量 $x$ 的后 $n-i-1$ 个分量,因此 np.dot(A[i, i+1:], x[i+1:]) 就是这两个向量的内积。它的计算公式为:

$\sum_{j=i+1}^{n-1} A_{ij} x_j$

其中,$A_{ij}$ 是矩阵 $A$ 的第 $i$ 行第 $j$ 列的元素,$x_j$ 是向量 $x$ 的第 $j$ 个分量。\\
Q:你可以帮我看我的代码哪里有问题吗?\\
\begin{lstlisting}
    def Jacobi(A, b, x_0, eps , max_iter):
    n = len(A)
    x = x_0.copy()

    for i in range(max_iter):
        x1 = np.zeros(n)
        for j in range(n):
            s1 = np.dot(A[j,:j],x1[:j])
            x1[j] = (b[j] - s1) / A[j, j]
        if np.allclose(x, x1, rtol=eps):
            return x1
        x = x1
    return x
\end{lstlisting}
A:\\
您的 Jacobi 方法代码看起来没有明显的问题,但是我注意到在计算 $s1$ 的时候,您的代码中写成了 np.dot(A[j,:j],x1[:j]),
这里应该是 np.dot(A[j,:j],x[:j]),因为 $x_1$ 还没有被更新。

另外,您使用了 np.allclose(x, x1, rtol=eps) 来判断两个向量是否相等。这里需要注意的是,allclose 函数比较的是两个向量之间的相对误差,而不是绝对误差。
这意味着,当两个向量的分量值都很小的时候,相对误差可能会比绝对误差更重要,因为分母比分子大很多。如果您更关心绝对误差,
可以使用 np.linalg.norm(x1 - x, ord=np.inf) < eps 来代替 allclose 函数。\\
Q:我们注意到jacobi方法的数学表示是$x^{(k+1)}_i  = \frac{1}{a_{ii}} \left(b_i -\sum_{j\ne i}a_{ij}x^{(k)}_j\right)$,
看起来只需要减去一个$\sum_{j\ne i}a_{ij}x^{(k)}_j$也就是我算法里面的a1,那为何我只减去了a1没有办法达成,
还需要再减去np.dot(A[j, j+1:], x[j+1:])才可以正常使用?\\
A:\\
您提到的Jacobi方法数学公式中确实只有一个$\sum_{j\ne i}a_{ij}x^{(k)}j$,但是在Python代码实现时,我们需要将其分解成两部分,即$a_1 = \sum{j=1}^{i-1} a_{ij}x^{(k)}j$和$a_2 = \sum{j=i+1}^{n} a_{ij}x^{(k)}_j$。

这是因为在Python中,使用numpy进行矩阵运算时,使用的是向量化运算,即一次性计算整个向量,而不是逐个元素计算。因此,我们需要将矩阵的某些部分单独取出来进行运算。

具体来说,np.dot(A[j, :i], x[:i])对应的是$a_1$,它计算的是矩阵A第j行前i个元素和向量$x$前i个元素的点积,对应的就是公式中的$\sum_{j=1}^{i-1} a_{ij}x^{(k)}j$;np.dot(A[j, i+1:], x[i+1:])对应的是$a_2$,它计算的是矩阵A第j行i+1到n个元素和向量$x$的i+1到n个元素的点积,对应的就是公式中的$\sum{j=i+1}^{n} a_{ij}x^{(k)}_j$。

因此,在实现Jacobi方法的Python代码中,需要将这两个部分分别计算并相加,才能得到正确的结果。



\section{123}
一根光纤中的模式频率可以使用波动方程计算:

$\nabla^2\vec{E}-\frac{1}{c^2}\frac{\partial^2\vec{E}}{\partial t^2} = 0$

其中,$\vec{E}$是电场,$c$是真空中的光速,$\nabla^2$是拉普拉斯算子。

假设电场可分为径向和方位角两个部分,则我们可以写成:

$\vec{E}(r,\theta,z,t)=\vec{E}(r)e^{-i\beta z}e^{i\omega t}\vec{u}(\theta)$

其中$r$是距离光纤轴线的径向距离,$\theta$是方位角度数,$z$是沿着光纤的距离,$\beta$ 是传播常数, $\omega $ 是角频率, $\vec {u } (\theta ) $ 是方位角单位矢量。

将这个表达式代入波动方程并简化后得到:

$
    \frac {1}{r}\frac {\partial }{\partial r}\left(r \frac {\partial \vec {E}}{\partial r }\right)+(\beta ^ 2-\frac {\omega ^ 2}{c ^ 2} n ^ 2 (r)) \ vec { E}=0
$

其中$n(r)$表示在径向距离$r $处的折射率。

该方程可以通过适当的边界条件求解,例如,在芯层-包层边界处电场及其导数的连续性。

模式频率与其角频率$\omega$有关:

$
    \omega =\frac {2\pi c}{\lambda}
$

其中,$\lambda $是光在光纤中的波长。

因此,模式频率可以表示为:

$
    f=\frac {\omega }{2\pi}=\frac {c}{\lambda}
$

其中$f$是模式的频率。

总之,在光纤中的模式频率与其波长有关,并且可以使用波动方程和适当的边界条件进行计算。该频率由光纤的折射率分布以及芯区域大小和形状决定。\\









\end{spacing}{}

\bibliographystyle{IEEEtran}
\bibliography{lecktion0}

\end{document}