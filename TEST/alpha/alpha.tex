\documentclass[12pt, a4paper, oneside]{article}
\usepackage{amsmath, amsthm, amssymb, graphicx}
\usepackage[bookmarks=true, colorlinks, citecolor=blue, linkcolor=black]{hyperref}
\usepackage[margin = 25mm]{geometry}
\usepackage{setspace}
\usepackage{listings}
\usepackage{cite}
\usepackage{ctex}
\usepackage{tabularx}
\usepackage{fancyhdr}
\title{Homework 0:Nobel Laureates Related to Lasers}
\date{\today}
\author{}
\begin{document}
\begin{spacing}{2.0}
\maketitle


\section{Nobel prizes in Physics}





弱全纯函数(weakly holomorphic function)是指在模形式理论中的一类函数,它们并不是全纯函数,但是在某些特定的条件下可以被视为是全纯函数的一种“推广”。

具体来说,对于一个离散子群 $\Gamma\subseteq \operatorname{SL}(2,\mathbb{Z})$,定义上半平面 $\mathcal{H}$ 上的函数 $f$,如果 $f$ 满足以下条件:

在 $\mathcal{H}$ 上的每个 $\gamma\in\Gamma$ 变换下,$f$ 变换成一个相差常数因子的函数,即 $f|{k}\gamma = (c{\gamma}z+d_{\gamma})^kf$,其中 $k$ 是一个自然数,$c_{\gamma},d_{\gamma}\in \mathbb{C}$,且 $c_{\gamma}d_{\gamma}^{-1}$ 为 $\gamma$ 的行列式值;

$f$ 在每个 $z\in \mathbb{Q}$ 处有一个“极点”,即存在一个正整数 $N$,使得 $(z-z_0)^N f(z)$ 在 $z_0$ 处有一个可去奇点,其中 $z_0$ 是一个有理数。

那么我们称 $f$ 是 $\Gamma$ 关于权为 $k$ 的弱全纯模形式(weakly holomorphic modular form)。通常情况下,弱全纯模形式被视为全纯模形式的一种“推广”,因为它们在某些重要的数学应用中也扮演着重要的角色。

\section{Nobel prizes in Physics}

在热平衡状态下,空腔中的自发辐射和受激辐射的功率密度之比为:

$\frac{W_s}{W_j} = \frac{g_1}{g_0}\exp\left(-\frac{h\nu}{kT}\right)$

其中,$W_s$ 是自发辐射功率密度,$W_j$ 是受激辐射功率密度,$g_1$ 和 $g_0$ 分别是激发态和基态的统计权重,$\nu$ 是辐射频率,$h$ 是普朗克常数,$k$ 是玻尔兹曼常数,$T$ 是温度。

对于可见光的波长 $\lambda = 0.5\ \mu\text{m}$,相应的频率 $\nu$ 可以通过光速和波长的关系求得:

$\nu = \frac{c}{\lambda} = \frac{3 \times 10^8\ \text{m/s}}{0.5 \times 10^{-6}\ \text{m}} = 6 \times 10^{14}\ \text{Hz}$

根据普朗克公式,辐射功率密度 $W$ 和辐射频率 $\nu$ 的关系为:

$W(\nu) = \frac{2h\nu^3}{c^2} \frac{1}{\exp\left(\frac{h\nu}{kT}\right)-1}$

将波长 $\lambda$ 对应的频率 $\nu$ 带入上式,得到该波长下的辐射功率密度为:

$W(\lambda) = \frac{2hc^2}{\lambda^5} \frac{1}{\exp\left(\frac{hc}{\lambda kT}\right)-1}$

因此,可见光波长为 $0.5\ \mu\text{m}$ 的自发辐射功率密度为:

$W_s(\lambda) = \frac{2hc^2}{\lambda^5} \frac{1}{\exp\left(\frac{hc}{\lambda kT}\right)-1}$

对于受激辐射功率密度,需要知道空腔内的受激辐射数密度 $n(\nu)$,即相同频率下的光子数密度。在热平衡状态下,$n(\nu)$ 与 $W(\nu)$ 的比值为:

$\frac{n(\nu)}{W(\nu)} = \frac{1}{h\nu/kT}$

因此,受激辐射功率密度为:
将自发辐射和受激辐射的功率密度代入比值公式中,得到:

$\frac{W_s}{W_j} = \frac{g_1}{g_0}\exp\left(-\frac{h\nu}{kT}\right) = \frac{g_1}{g_0}\exp\left(-\frac{hc}{\lambda kT}\right)$

其中,$g_1$ 和 $g_0$ 分别为激发态和基态的统计权重,需要根据具体系统来确定。对于热平衡的黑体辐射,所有的能级都被充分激发,$g_1$ 和 $g_0$ 均为简并度,即:

$g_1 = g_0 = \frac{8\pi V}{h^3} \left(\frac{2\pi m kT}{h^2}\right)^{3/2}$

其中,$V$ 是空腔体积,$m$ 是辐射粒子的质量。

将数值代入比值公式中,得到:

$\frac{W_s}{W_j} = \exp\left(-\frac{hc}{\lambda kT}\right) = \exp\left(-\frac{6.626 \times 10^{-34}\ \text{J}\cdot\text{s} \times 3 \times 10^8\ \text{m/s}}{0.5 \times 10^{-6}\ \text{m} \times 1.38 \times 10^{-23}\ \text{J/K} \times 1500\ \text{K}}\right) \approx 4.82 \times 10^{-11}$

因此,可见光波长为 $0.5\ \mu\text{m}$ 的自发辐射功率密度与受激辐射功率密度之比约为 $4.82 \times 10^{-11}$。
\section{3}

考虑一个具有面积$A$的平面区域,其法线方向为$\mathbf{n}$,表面上的任意一点与法线的夹角为$\theta$,如下图所示:

假设该表面是一个完美的余弦辐射体,其辐射出射度为$M_{\lambda}$,单位为$W/(m^2\cdot sr\cdot nm)$。我们需要推导出它的表达式。

首先考虑一个单位面积上的辐射能量。由基本电磁理论可知,一个振荡频率为$\nu$、振幅为$E$的电磁波在介质中的能流密度为:
其中,$\epsilon_0$是真空介电常数,$c$是光速。

根据辐射出射度的定义,单位面积上在某一频率范围内沿某一方向辐射的能量为:

�
�
=
�
�
cos
⁡
�
�
�
⋅
�
�
⋅
�
�
dE=M 
λ
​
 cosθdA⋅dλ⋅dt
其中,$d\lambda$是波长的微元,$dt$是时间的微元。这里将$dA$乘以$\cos{\theta}$是因为辐射方向与表面法线的夹角为$\theta$,只有与该方向垂直的电磁波才能被视为真正的辐射。

假设表面上存在一个微元$dA_0$,其法线方向为$\mathbf{n_0}$,与辐射方向的夹角为$\alpha$。根据余弦定理可知:

其中,$\cos{\beta} = \cos{\theta_0}$,$\sin{\beta} = \sqrt{1 - \cos^2{\theta_0}}$,$\cos{\gamma} = \cos{\phi_0 - \phi}$,$\theta_0$和$\phi_0$是微元$dA_0$法线的天顶角和方位角,$\phi$是辐射方

接下来,我们需要将$d\Omega$用角度$\theta$和$d\theta$表示出来。如下图所示,假设我们关注单位时间内,从立体角$\Omega$内某一点$P$向外发射的能量,设该点与坐标系原点的距离为$r$。


显然,对于单位面积的投影面积$ds$,$P$点到该面积的距离为$r\cos\theta$。因此,该面积接收到的来自$P$点的辐射功率为:

$$ dP = \frac{dE}{dt}\cdot ds = \frac{dE}{dt}\cdot r^2\cos\theta d\Omega $$

其中,$d\Omega$表示单位面积在球面上对应的立体角元素,它与$d\theta$和$d\phi$相关,具体地:

$$ d\Omega = \sin\theta d\theta d\phi $$

将$d\Omega$代入上式得:

$$ dP = \frac{dE}{dt}\cdot r^2\cos\theta \sin\theta d\theta d\phi $$

将$dE$代入上式得:

$$ dP = \frac{q^2}{6\pi\epsilon_0c^3}\beta^2\sin^2\theta \cos\theta d\theta d\phi $$

最终,我们得到了余弦辐射体的辐射出射度表达式:

$$ \frac{dP}{d\Omega} = \frac{q^2}{6\pi\epsilon_0c^3}\beta^2\sin^2\theta \cos\theta $$

其中,$\beta$为辐射体的速度与光速的比值



\section{11}
朗伯辐射体是指一个均匀的热辐射表面,它的辐射出射度表达式为:

$$ \frac{dP}{d\Omega} = \frac{2\pi k_B^2 T^4}{h^3c^2}\cos\theta $$

其中,$dP$为单位时间内辐射在$\Omega$立体角内的功率,$d\Omega$为立体角元素,$k_B$为玻尔兹曼常数,$T$为表面的温度,$h$为普朗克常数,$c$为光速,$\theta$为法向量与$\mathbf{n}$的夹角。

为了推导上述表达式,我们假设辐射体是均匀、各向同性的。考虑一个面积为$A$的小面元$dS$,该小面元在$\mathbf{n}$方向上的投影面积为$dS\cos\theta$。设辐射体在频率为$\nu$到$\nu+d\nu$的波段内辐射的平均能量密度为$u(\nu)$,则该小面元在单位时间内辐射出的功率为:

$$ dP = A dS\cos\theta u(\nu) c $$

将普朗克黑体辐射公式代入上式,得到:

$$ dP = \frac{2\pi h\nu^3}{c^2}\frac{1}{e^{\frac{h\nu}{k_BT}}-1}A dS\cos\theta $$

根据能量守恒,辐射出的能量应等于该小面元吸收的能量。假设该小面元在频率为$\nu$到$\nu+d\nu$的波段内吸收的平均能量密度为$u'(\nu)$,则有:

$$ dP = A dS\cos\theta u'(\nu) c $$

将普朗克黑体辐射公式代入上式,得到:

$$ dP = \frac{2\pi h\nu^3}{c^2}\frac{1}{e^{\frac{h\nu}{k_BT}}-1}A dS\cos\theta $$

将两式相等,得到:

$$ u(\nu) = u'(\nu) $$

因此,辐射体在所有方向上的辐射能量密度$u(\nu)$与吸收能量密度$u'(\nu)$相等,且与方向无关。将辐射能量密度代入辐射出射度表达式中,得到:

$$ \frac{dP}{d\Omega} = \frac{2\pi}{c}u(\nu)\cos\theta = \frac{2\pi k_B^2 T^4}{h^3c^2}\cos\theta $$

其中,$u(\nu)$为普朗克黑体辐射公式中的辐

根据朗伯体的定义,表面每个微小立体角$d\Omega$上的辐射强度相等,即$I(\theta,\phi)=I_0$,其中$I_0$为表面上某一点的辐射强度,因此有:

$$dI(\theta,\phi)=I_0 \cos\theta \cdot d\Omega=I_0 \cos\theta \cdot \sin\theta d\theta d\phi$$

其中$d\Omega$可以用$\sin\theta d\theta d\phi$表示,代入普朗克黑体辐射公式,有:

$$dE=\frac{h\nu^3}{c^2} \cdot \frac{1}{e^{h\nu/kT}-1} \cdot \cos\theta \cdot \sin\theta d\theta d\phi d\nu A$$

对$dE$在球坐标系下进行积分,可得总辐射能量$E$:

$$E=\int_0^{2\pi}\int_0^{\frac{\pi}{2}}\int_0^{\infty}\frac{h\nu^3}{c^2} \cdot \frac{1}{e^{h\nu/kT}-1} \cdot \cos\theta \cdot \sin\theta d\theta d\phi d\nu A$$

化简并代入普朗克黑体辐射公式中的积分,有:

$$E=\frac{4\pi k^4}{c^2h^3}\cdot T^4\cdot A$$

将表面积$A$代入,得到表面单位面积上的辐射能量为:

$$j^*=E/A=\frac{4\pi k^4}{c^2h^3}\cdot T^4$$

其中$j^*$即为朗伯辐射体的辐射出射度表达式。\\
根据定义,光通量$F$和光强度$I$之间的关系为:

$$F = K_m \int_{\lambda_1}^{\lambda_2} V(\lambda) I(\lambda) d\lambda$$

其中,$K_m$为光通量系数,其值为683 lm/W,$\lambda_1$和$\lambda_2$分别为可见光的最短波长和最长波长,分别为400 nm和700 nm。

将已知条件代入上式,得到:

$$100 \text{ lm} = 683 \text{ lm/W} \cdot 0.50 \cdot I(0.510 \mu \text{m}) \cdot 10^3 \text{ W/m}^2$$

解得波长为$0.510 \mu \text{m}$的绿光的光强度为:

$$I(0.510 \mu \text{m}) = \frac{100 \text{ lm}}{683 \text{ lm/W} \cdot 0.50 \cdot 10^3 \text{ W/m}^2} = 0.292 \text{ W/m}^2$$

将该光强度代入普朗克-爱因斯坦定理,可得该波长下单位面积的辐射能量为:

$$E = h \nu \cdot \frac{I(0.510 \mu \text{m})}{c} = \frac{h c}{\lambda} \cdot I(0.510 \mu \text{m}) = 2.246 \times 10^{-19} \text{ J/m}^2$$

因此,在1 min时间内,屏幕接受的辐射能量为:

$$E_{\text{total}} = E \cdot A \cdot t = 2.246 \times 10^{-19} \text{ J/m}^2 \cdot A \cdot 60 \text{ s}$$

其中,$A$为屏幕的面积。

根据普朗克-爱因斯坦关系$E=h\nu$,其中$h$为普朗克常数,$\nu$为光子的频率。由于给定的是波长$\lambda$,可以使用$c=\lambda \nu$将其转换为频率$\nu$,其中$c$为光速。因此,可以将$E=h\nu$重写为$E=\frac{hc}{\lambda}$。

题目中给出绿光波长为$0.510\mu\text{m}$,代入上式可得$E=\frac{hc}{\lambda}=2.441\times10^{-19}\text{ J}$。题目中还给出绿光的光通量为$100\text{ lm}$,因此绿光的辐射照度$I(\lambda)$为$I(\lambda)=\frac{100\text{ lm}}{\text{m}^2}=100\text{ lx}$,其中$1\text{ lx}=1\text{ lm/m}^2$。

根据辐射照度$I(\lambda)$的定义,可以得到在波长为$\lambda$的光的垂直入射面上的光辐射功率密度$P(\lambda)$为$P(\lambda)=I(\lambda)V(\lambda)$,其中$V(\lambda)$为视见函数。题目中给出绿光的视见函数为$V(0.510\mu\text{m})=0.50$。

因此,波长为$0.510\mu\text{m}$的绿光在垂直入射面上的光辐射功率密度$P(\lambda)$为$P(\lambda)=I(\lambda)V(\lambda)=100\text{ lx} \cdot 0.50=50\text{ W/m}^2$。

最后,根据辐射能量密度的定义$E=P(\lambda)\cdot t$,其中$t$为时间,可以求出在1 min时间内屏幕接受的辐射能量为$E=P(\lambda)\cdot t=50\text{ W/m}^2 \cdot 60\text{ s}=3000\text{ J/m}^2=3\text{ mJ/m}^2$。

将波长为$0.510 \mu m$的绿光的辐射能量密度代入可得$E = \frac{hc}{\lambda} \cdot I(0.510 \mu \text{m}) = 2.246 \times 10^{-19} \text{ J/m}^2$。


$$u(\lambda, T) = \frac{8\pi hc}{\lambda^5} \cdot \frac{1}{\mathrm{e}^{\frac{hc}{\lambda k_B T}}-1}$$




根据光的频率和波长之间的关系$f=c/\lambda$,其中$c$是光速,可以计算出相应的频率:

对于中心波长为$0.5 \mu m$的光波,其频率为$f=c/\lambda=3\times 10^8 \text{m/s} / (0.5\times 10^{-6} \text{m})=6\times 10^{14} \text{Hz}$,而谱线宽度为$1 nm$,因此其频率宽度可以通过以下公式计算:

$$\Delta f = \frac{\Delta \lambda}{\lambda^2}c = \frac{1\times 10^{-9} \text{m}}{(0.5\times 10^{-6} \text{m})^2}3\times 10^8 \text{m/s} = 12 \times 10^{12} \text{Hz}$$

因此,中心波长为$0.5 \mu m$、谱线宽度为$1 nm$的光波的频率宽度为$12 \times 10^{12} \text{Hz}$。

对于中心波长为$1 \mu m$的光波,其频率为$f=c/\lambda=3\times 10^8 \text{m/s} / (1\times 10^{-6} \text{m})=3\times 10^{14} \text{Hz}$,而谱线宽度为$1 nm$,因此其频率宽度可以通过以下公式计算:

$$\Delta f = \frac{\Delta \lambda}{\lambda^2}c = \frac{1\times 10^{-9} \text{m}}{(1\times 10^{-6} \text{m})^2}3\times 10^8 \text{m/s} = 3 \times 10^{12} \text{Hz}$$

因此,中心波长为$1 \mu m$、谱线宽度为$1 nm$的光波的频率宽度为$3 \times 10^{12} \text{Hz}$。

对于中心波长为$10 \mu m$的光波,其频率为$f=c/\lambda=3\times 10^8 \text{m/s} / (10\times 10^{-6} \text{m})=3\times 10^{13} \text{Hz}$,而谱线宽度为$1 nm$,因此其频率宽度可以通过以下公式计算:

$$\Delta f = \frac{\Delta \lambda}{\lambda^2}c = \frac{1\times 10^{-9} \text{m}}{(10\times 10^{-6} \text{m})^2}3\times 10^8 \text{m/s} = 3 \times 10^{10} \text{Hz}$$

因此,中心波长为$10 \mu m$、谱线宽度为$1 nm$的光波的频率宽度为$3 \times 10^{10} \text{Hz}$。













\end{spacing}{}

\bibliographystyle{IEEEtran}
\bibliography{lecktion0}

\end{document}

