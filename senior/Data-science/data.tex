\documentclass[12pt, a4paper, oneside]{article}
\usepackage{amsmath, amsthm, amssymb, graphicx}
\usepackage[bookmarks=true, colorlinks, citecolor=blue, linkcolor=black]{hyperref}
\usepackage[margin = 25mm]{geometry}
\usepackage{setspace}
\usepackage{listings}
\usepackage{ctex}
\usepackage{cite}
\usepackage{color}
\usepackage{float}

\title{}
\date{\today}
\author{孙陶庵 202011010101}
\begin{document}
\begin{spacing}{2.0}
\tableofcontents
\maketitle


\section{背景介绍及数据分类}
\subsection{背景介绍}
The purpose of making the dataset is to prove that the stars follows a certain graph in the celestial Space ,
specifically called Hertzsprung-Russell Diagram or simply HR-Diagram
so that we can classify stars by plotting its features based on that graph.\\
\\
Data Collection and Preparation techniques:
The dataset is created based on several equations in astrophysics. They are given below:\\
\\
Stefan-Boltzmann's law of Black body radiation (To find the luminosity of a star)\\
Wienn's Displacement law (for finding surface temperature of a star using wavelength)\\
Absolute magnitude relation\\
Radius of a star using parallax .\\
The dataset took 3 weeks to collect for 240 stars which are mostly collected from web.\\
The missing data were manually calculated using those equations of astrophysics given above.\\


\subsection{数据分类}


数据文件包含以下元素,分别为:
1. Temperature (K):温度(Kelvin)\\
2. Luminosity(L/Lo):亮度\\
3. Radius(R/Ro):半径;天文学中会使用“太阳半径”(Solar radius),通常表示为R/Ro,其中Ro是太阳的半径。这个比率将每颗恒星的半径与太阳的半径进行比较,使得比较更为标准化。\\
4. Absolute magnitude(Mv):绝对星等;是一个天体的亮度,以一定的距离为标准,用于消除距离对亮度的影响。\\
5. Star type:Red Dwarf, Brown Dwarf, White Dwarf, Main Sequence , SuperGiants, HyperGiants分别为红矮星、褐矮星、
白矮星、主序星、超巨星、特超巨星\\
6. Star color:星体颜色\\
7. Spectral Class:星体光谱;一般会从高温到低温分为OBAFGKM分成七个主要类别

\section{理论基础}
\subsection{温度与颜色关系} 
温度 (Temperature) 和星星的颜色 (Star color) 之间可能存在关联。通常,更高温度的恒星呈现蓝色,而较低温度的恒星呈现红色。\\
1.光谱分类: 恒星的光谱类型是根据它们的光谱特征来分类的,通常采用哈佛光谱分类系统,从高温到低温分为OBAFGKM七个主要类别。这个分类与恒星的表面温度有关,高温星体的光谱呈现蓝色,低温星体呈现红色。\\
2.黑体辐射定律: 黑体辐射定律描述了一个理想的黑体(完美吸收所有入射辐射的物体)的辐射强度分布。恒星的表面温度决定了它的光谱,因为星体的辐射与其表面温度有关。
\subsection{亮度和距离} 
亮度与绝对星等关系: 亮度 (Luminosity) 和绝对星等 (Absolute magnitude) 之间可能存在关联。
这两个参数通常与星星的亮度有关。亮度是一个物体发光的强度,通常用光度(Luminosity)表示。
绝对星等(Absolute magnitude)则是一个天体的亮度,但是以一定的距离为标准,用于消除距离对亮度的影响。
这两个参数通常在天文学中用来描述天体的亮度水平。\\
1.目视亮度(视星等):即以地球为观察点测得的星等,以m表示\\
2.绝对亮度(绝对星等):从距离星体10个秒差距(32.6光年)的地方看到的目视亮度(也就是视星等),叫做该星体的绝对星等以M表示\\
3.$M=m+5-\log\left(\frac{d_0}{d}\right)$,$d$为恒星距离(秒差距),$d_0$ 为一秒差距\\
4.这个实验以此理论为处理数据的依据\\
5.赫罗图: 赫罗图是一种将亮度和温度(或颜色指数)相对应的图表,用于展示不同类型的星体在这个图上的分布。主序带是赫罗图上最为明显的特征,显示了主序星的分布。
\subsection{半径与亮度关系}
半径 (Radius) 和亮度 (Luminosity) 之间可能存在一些关联。更大的恒星可能更亮,但也可能有其他因素影响这种关系。你可以绘制一个半径和亮度的散点图,以查看它们之间的关系。\\
1.半径与亮度关系: 恒星的半径与其亮度之间存在关系,这通常涉及到恒星的物理特性。较大的恒星通常具有更高的亮度,但具体的关系也受到其他因素的影响,比如温度和光度。\\
2.半径与亮度的物理原理: 根据斯特凡-玻尔兹曼定律,恒星的亮度(Luminosity)与其表面温度的四次方成正比,而表面积(与半径平方成正比)也会影响亮度。这表明半径与亮度之间的关系并非线性。\\
3.恒星分类与其他参数的关系: 恒星分类 (Star type) 与温度、亮度等参数之间可能存在一些模式。你可以使用统计方法或绘制图表来研究这些关系。
\subsection{恒星光谱分类} 
恒星的光谱分类是根据它们的光谱特征进行的,通常使用哈佛光谱分类系统,将恒星分为七个主要类别:O、B、A、F、G、K、M。这个分类与恒星的温度和表面特性有关,高温星体光谱呈蓝色,低温星体呈红色。\\
1.恒星分类与温度关系: 恒星的光谱类型与其表面温度有密切关系。通常来说,O型恒星温度较高,而M型恒星温度较低。
\subsection{光谱类型与其他参数的关系} 
光谱类型 (Spectral Class) 通常与温度相关。你可以研究不同光谱类型的恒星的温度分布,并查看其他参数是否与光谱类型有关。\\
1.光谱类型与温度: 恒星的光谱类型通常与其表面温度有关。恒星的光谱类型按照OBAFGKM的顺序,从高温到低温排列。你可以创建一个箱线图或直方图,将光谱类型划分为不同的组别,然后观察每个组别中温度的分布情况。\\
2.光谱类型与亮度: 不同光谱类型的恒星可能具有不同的亮度。通过绘制光谱类型和亮度之间的散点图,你可以看到它们之间是否存在某种趋势或模式。\\
3.光谱类型与半径: 光谱类型也可能与恒星的半径有关。通过绘制光谱类型和半径之间的散点图,你可以研究它们之间的关系,看看是否存在某种相关性。\\
4.光谱类型与星星的颜色: 光谱类型和星星的颜色之间可能存在某种对应关系。你可以创建一个交叉表或堆叠条形图,将光谱类型和星星颜色进行比较,以便更好地理解它们之间的联系。

\section{代码思路}
\subsection{温度与颜色关系}
1.1.根据温度生成Wiens位移定律的图形。\\
1.2.根据给定数据集中的温度和颜色信息,画出恒星的颜色分布。\\
1.3.将这两个图形绘制在一张图上进行比较。\\
\\
2.亮度与绝对星等关系\\
2.1.散点图可视化: 使用散点图来展示亮度和绝对星等之间的关系。每个点代表一个星体,横轴是绝对星等,纵轴是亮度。这样的图表有助于观察数据的分布趋势。\\
2.2.相关性分析: 使用相关性系数来量化半径和亮度之间的线性关系。相关性系数越接近1或-1,表示两者之间的关系越强。\\
2.3.赫罗图绘制: 有温度或颜色指数的数据,可以绘制赫罗图,将绝对星等和温度进行对应,以显示不同类型星体的分布。\\
\\
3.半径与亮度关系\\
3.1.散点图可视化: 使用散点图来直观地展示半径和亮度之间的关系。每个点代表一个恒星,横轴是半径,纵轴是亮度。\\
3.2.相关性分析: 使用相关性系数来量化半径和亮度之间的线性关系。相关性系数越接近1或-1,表示两者之间的关系越强。\\
3.3.分组分析: 如果你有其他分类信息,比如恒星的类型,可以尝试按照这些类型进行分组,然后绘制每个组的半径与亮度的关系图,以便更详细地了解不同类型恒星之间的差异。\\
\\
4.恒星分类与其他参数的关系\\
4.1.数据分组: 将数据按照恒星分类进行分组。\\
4.2.探索性分析: 针对每个恒星分类,分析温度、亮度、半径等参数的分布情况。你可以使用直方图、箱线图或核密度图来可视化这些分布。\\
4.3.相关性分析: 使用统计方法(如相关系数)来量化不同参数之间的关联程度。你可以计算不同恒星分类下温度、亮度、半径之间的相关系数。\\
\\
5.光谱类型与其他参数的关系\\
5.1.光谱类型与温度关系的散点图: 使用散点图展示光谱类型与温度之间的关系。横轴是光谱类型,纵轴是温度。你可以使用不同颜色或标记不同的光谱类型,以更清晰地显示数据。\\
5.2.光谱类型和温度的箱线图: 通过箱线图来展示不同光谱类型下温度的分布情况。这有助于观察每个光谱类型中温度的变化范围和分布情况。\\




\end{spacing}{}
%\bibliographystyle{IEEEtran}
%\bibliography{re1}

\end{document}