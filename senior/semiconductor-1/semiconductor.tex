\documentclass[12pt]{article}
\usepackage[a4paper, left=3.17cm, right=3.17cm, top=2.54cm, bottom=2.54cm]{geometry}
\usepackage[T1]{fontenc}
\usepackage{mathptmx}
\usepackage{amsmath, amsthm, amssymb, graphicx}
\usepackage{amsfonts}
\usepackage{chemformula}
\usepackage{cite}
\usepackage{ctex}
\usepackage{float}
\usepackage{listings}
\usepackage{setspace}
\usepackage[colorlinks, linkcolor=black, anchorcolor=black, citecolor=black]{hyperref}

\definecolor{codegreen}{rgb}{0,0.6,0}
\definecolor{codegray}{rgb}{0.5,0.5,0.5}
\definecolor{codepurple}{rgb}{0.58,0,0.82}
\definecolor{backcolour}{rgb}{0.95,0.95,0.92}

\lstdefinestyle{mystyle}{
    backgroundcolor=\color{backcolour},   
    commentstyle=\color{codegreen},
    keywordstyle=\color{magenta},
    numberstyle=\tiny\color{codegray},
    stringstyle=\color{codepurple},
    basicstyle=\ttfamily\footnotesize,
    breakatwhitespace=false,         
    breaklines=true,                 
    captionpos=b,                    
    keepspaces=true,                 
    numbers=left,                    
    numbersep=5pt,                  
    showspaces=false,                
    showstringspaces=false,
    showtabs=false,                  
    tabsize=2
}

\lstset{style=mystyle}

\setlength{\parskip}{0.5em}
\doublespacing
\title{Quartus II、Modelsim、FPGA操作入门}
\author{第一组 物理2001孙陶庵 202011010101}
\begin{document}
\def\hbar{\text{\textit{ħ}}}

    %——————————————————————————————————设置公式编号———————————————————————————————————————%
    \numberwithin{equation}{subsection}
    \renewcommand\theequation{\thesubsection.\arabic{equation}}
    %%%%%%%%%%%%%%%%%%%%%%%%%%%%%%%%%%%%%%%%%%%%%%%%%%%%%%%%%%%%%%%%%%%%%%%%%%%%%%%%%%%%%%
    \newtheorem{mythm}{定理}[section]  %定义一个定理环境
    %\maketitle  

\begin{titlepage}
	\newcommand{\HRule}{\rule{\linewidth}{0.5mm}}
	\includegraphics[width=12.5cm]{sc_logo.jpg}\\[0cm] 
	\center 
	\quad\\[1.5cm]
	\textsl{\Large 集成电路实验一 }\\[0.5cm] 
	\textsl{\large 数字系统综合实验}\\[0.5cm] 
	\makeatletter
	\HRule \\[0.4cm]
	{ \huge \bfseries \@title}\\[0.4cm] 
	\HRule \\[1.5cm]
	\begin{minipage}{0.4\textwidth}
		\begin{flushleft} \large
			\emph{报告人}\\
			\@author 
		\end{flushleft}
	\end{minipage}
	~
	\begin{minipage}{0.4\textwidth}
		\begin{flushright} \large
			\emph{实验时间地点} \\
			\textup{2023/12/05\\物电院A116}
		\end{flushright}
	\end{minipage}\\[3cm]
	\makeatother
	{\large \emph{集成电路实验一}}\\[0.5cm]
	{\large \today}\\[2cm] 
	\vfill 
\end{titlepage}
\thispagestyle{empty}
\clearpage     %另起一页  
\pagenumbering{roman}
\tableofcontents        %输出目录
\clearpage     %另起一页 


\pagestyle{headings}
\pagenumbering{arabic}


\section{实验原理}
\section{实验内容}
\subsection{实验步骤}
使用Quartus II编程,写出一份可以操控电路板上led灯的程序,并调控参数观察现象。\\
实验步骤:\\
1.	电脑创建程序写出程序\\
2.	编译,debug\\
3.	使用Modelsim 仿真并观察波形, 确认无误后将电路板接进电脑\\
4.	使用Quartus II 将程序导入电路板\\

\subsection{实验结果}
实验中使用的代码:
1.
\begin{lstlisting}

module led (input wire clk,input wire rst_n,output  reg[3:0] led);

reg [25:0] cnt_1s;    //1秒计数器

parameter END_CNT_1S = 499999999;  //1秒计数结束标识

always @(posedge clk or negedge rst_n)
	if(!rst_n)
	cnt_1s <='d0;
	else if(cnt_1s==END_CNT_1S)
	cnt_1s <='d0;
	else cnt_1s <= cnt_1s+1'b1;
always @(posedge clk or negedge rst_n)
	if(!rst_n)
	led <= 4'b0000;
	else if(cnt_1s==END_CNT_1S)
		case(led)
		4'b0001:  led<=4'b1000;
		4'b0010:  led<=4'b0001;
		4'b0100:  led<=4'b0010;
		4'b1000:  led<=4'b0100;
		default:  led<=4'b0000;
		endcase
	else led <= led;
endmodule
\end{lstlisting}
$cnt\_1s$ 是一个 26 位的计数器,用于计算时钟信号 clk 的上升沿。在正常运行时,计数器逐次递增,表示经过的时钟周期数。

$END\_CNT\_1S$ 是一个参数,它定义了计数器的结束标识。在这个例子中,计数器 $cnt\_1s$ 的结束标识是 $END\_CNT\_1S$,当计数器达到这个值时,将重新计数。

$always @(posedge\ clk\ or\ negedge\ rst\_n)$ 语句块用于处理时钟和异步复位信号。在时钟的上升沿或异步复位信号 $rst\_n$ 的下降沿触发时执行相应的逻辑。

在第一个 always 块中,计数器 $cnt\_1s$ 在每个时钟周期上升沿时递增。在异步复位信号 $rst\_n$ 的下降沿时,计数器被清零。

在第二个 always 块中,LED状态机的逻辑被实现。当异步复位信号 $rst\_n$ 的下降沿发生时,LED被初始化为全灭状态 (4'b0000)。
在计数器 $cnt\_1s$ 达到结束标识 $END\_CNT\_1S$ 时,LED状态根据一个简单的状态机进行变化。LED在四种状态之间循环切换,实现了一个基本的流水灯效果。
\begin{lstlisting}

`timescale 1ns/1ns
module tb_led;
	reg	clk;
	reg	rst_n;
	wire[3:0] led;
	
	led led_inst(
			.clk	(clk),
			.rst_n (rst_n),
			.led	(led)
			);
always
	begin
		#10;
		clk=~clk;
	end
initial
	begin
		clk	=1'b1;
		rst_n=1'b0;
		#50
		rst_n=1'b1;
		#400000000;
		$stop;
	end
endmodule
\end{lstlisting}
timescale 1ns/1ns:这是一个时间刻度(timescale)声明,指定了时序仿真的时间单位和时间精度。

module $tb\_led$;:定义一个模块 $tb\_led$,即测试台模块。

reg clk; 和 reg $rst\_n$;:声明时钟信号 clk 和异步复位信号 $rst\_n$,这些信号将用于测试。

wire [3:0] led;:声明一个 4 位宽的 led 信号,用于连接被测试的 led 模块的输出。

led $led\_inst$(...);:实例化被测试的 led 模块,并连接输入输出信号。

always begin ... end:一个无限循环的 always 块,用于模拟时钟信号 clk 的上升沿。

$\#$10;:表示延迟 10 个时间单位,模拟时钟周期。
clk =~clk;:在每个时钟周期的上升沿和下降沿交替改变 clk 的值。
initial begin ... end:初始块,用于在仿真开始时初始化测试环境和控制仿真流程。

clk = 1'b1;:初始时钟信号为高电平。
$rst\_n$ = 1'b0;:初始时,异步复位信号为低电平。
$\#$50:等待一段时间,模拟异步复位信号的持续时间。
$rst\_n$ = 1'b1;:将异步复位信号置为高电平,启动 LED 模块的正常操作。
$\#$400000000;:等待足够的时间,以确保仿真足够长以捕获 LED 模块的运行情况。
$\$$stop;:停止仿真。
这个测试台通过时钟信号和异步复位信号对 led 模块进行仿真测试,可以观察模块在仿真时的行为。







\subsection{拓展实验}
读懂代码,试修改代码,完成以下实验步骤。\\
a.	如何改变流水灯的移动速度?\\
b.	如何改变流水灯的移动方向?\\
c.	如何移动两个LED灯?\\
d.	分别用$Key\_1$、 $Key\_2$、 $Key\_3$三个按键来控制移动频率1Hz、2Hz和移动方向\\
e.	查看资源占有率及RTL视图\\
拓展实验结果:\\
a.LED的移动速度是通过计时器 counter 和计时结束标识 $end\_cnt$ 来控制的。所以可以透过修改这一部分来改变移动速度
\begin{lstlisting}
    always @(posedge clk or negedge rst_n or negedge key_1 or negedge key_2)
    if (!rst_n)
        counter <= 0;
        end_cnt <= END_CNT_1S;  // 初始设置为1秒的计时结束标识
    else if (!key_1)  // 按下后周期变为1S
        end_cnt <= END_CNT_1S;
    else if (!key_2)  // 按下后周期变为2S
        end_cnt <= END_CNT_2S;
    else if (counter == end_cnt)
        counter <= 0;
    else 
        counter <= counter + 1'b1;

\end{lstlisting}

通过按下 $Key\_1$,LED的速度将被设置为1秒一次,而按下$Key\_2$,LED的速度将被设置为2秒一次。这是因为 $end\_cnt$ 在这里用于确定计时器何时达到结束状态,进而改变LED的状态。

因此,如果希望改变LED的移动速度,可以通过按下不同的按键来选择不同的计时结束标识。\\

b.LED的移动方向是通过按键输入 $key\_3$ 控制的。按下 $key\_3$ 时,direction寄存器会取反,从而改变LED的移动方向。

\begin{lstlisting}

always @(posedge clk or negedge rst_n or negedge key_1 or negedge key_2 or negedge key_3)
    if (!rst_n)
    begin
        counter <= 0;
        end_cnt <= END_CNT_1S;  // 初始设置为1秒的计时结束标识
        direction <= 1'b0;  // 初始方向为0
    end
    else if (!key_1)  // 按下后周期变为1S
        end_cnt <= END_CNT_1S;
    else if (!key_2)  // 按下后周期变为2S
        end_cnt <= END_CNT_2S;
    else if (!key_3)  // 按下后改变方向
        direction <= ~direction;  // 取反方向
    else if (counter == end_cnt)
        counter <= 0;
    else 
        counter <= counter + 1'b1;

\end{lstlisting}
当按下 $key\_3$ 时,direction 的值将翻转,从而改变LED的移动方向。这是通过 direction <= ~direction; 这一语句实现的。

因此,如果希望改变LED的移动方向,只需按下 $key\_3$ 即可。

c.
可以尝试使用两个计时器和两个计时结束标识来分别控制两个LED的移动。
\begin{lstlisting}

module dual_led (
    input wire clk,
    input wire rst_n,
    input wire key_1,
    input wire key_2,
    input wire key_3,
    output reg[3:0] led1,
    output reg[3:0] led2
);

reg [25:0] counter1;    // 1秒计数器1
reg [25:0] counter2;    // 1秒计数器2
reg [25:0] end_cnt1;    // 结束时间周期1
reg [25:0] end_cnt2;    // 结束时间周期2
reg direction1;         // 方向1,0左1右
reg direction2;         // 方向2,0左1右

parameter END_CNT_1S = 4999999;  // 1秒计数结束标识
parameter END_CNT_2S = 9999999;  // 2秒计数结束标识

always @(posedge clk or negedge rst_n or negedge key_1 or negedge key_2 or negedge key_3)
    if (!rst_n)
    begin
        counter1 <= 0;
        end_cnt1 <= END_CNT_1S;
        direction1 <= 1'b0;  // 初始方向为0
    end
    else if (!key_1)
        end_cnt1 <= END_CNT_1S;
    else if (!key_2)
        end_cnt1 <= END_CNT_2S;
    else if (!key_3)
        direction1 <= ~direction1;

    // Counter1 logic
    if (counter1 == end_cnt1)
        counter1 <= 0;
    else
        counter1 <= counter1 + 1'b1;

always @(posedge clk or negedge rst_n or negedge key_1 or negedge key_2 or negedge key_3)
    if (!rst_n)
    begin
        counter2 <= 0;
        end_cnt2 <= END_CNT_1S;
        direction2 <= 1'b0;  // 初始方向为0
    end
    else if (!key_1)
        end_cnt2 <= END_CNT_1S;
    else if (!key_2)
        end_cnt2 <= END_CNT_2S;
    else if (!key_3)
        direction2 <= ~direction2;

    // Counter2 logic
    if (counter2 == end_cnt2)
        counter2 <= 0;
    else
        counter2 <= counter2 + 1'b1;

always @(posedge clk or negedge rst_n)
begin
    if (!rst_n)
    begin
        led1 <= 4'b0001;
        led2 <= 4'b0001;
    end
    else
    begin
        // LED1 logic
        case({counter1 == end_cnt1, direction1})
            2'b10: led1 <= led1 << 1;
            2'b01: led1 <= led1 >> 1;
            default: led1 <= led1;
        endcase

        // LED2 logic
        case({counter2 == end_cnt2, direction2})
            2'b10: led2 <= led2 << 1;
            2'b01: led2 <= led2 >> 1;
            default: led2 <= led2;
        endcase
    end
end

endmodule
\end{lstlisting}
分别由两个计时器 counter1 和 counter2 控制。按键 $key\_3$ 仍然用于改变方向,
而按键 $key\_1$ 和 $key\_2$ 分别用于选择1秒和2秒的计时结束标识。LED的移动逻辑分别由两个独立的 case 语句控制。\\
d.
$Key\_1$ 选择1Hz的计时结束标识和向左移动。
$Key\_2$ 选择2Hz的计时结束标识和向左移动。
$Key\_3$ 按下则切换移动方向。











%\bibliographystyle{IEEEtran}
%\bibliography{qp}

\end{document}
