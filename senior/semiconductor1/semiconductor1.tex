\documentclass[12pt, a4paper, oneside]{article}
\usepackage{amsmath, amsthm, amssymb, graphicx}
\usepackage[bookmarks=true, colorlinks, citecolor=blue, linkcolor=black]{hyperref}
\usepackage[margin = 25mm]{geometry}
\usepackage{setspace}
\usepackage{listings}
\usepackage{ctex}
\usepackage{cite}
\usepackage{color}
\usepackage{float}
\usepackage{xcolor}
\definecolor{codegreen}{rgb}{0,0.6,0}
\definecolor{codegray}{rgb}{0.5,0.5,0.5}
\definecolor{codepurple}{rgb}{0.58,0,0.82}
\definecolor{backcolour}{rgb}{0.95,0.95,0.92}

\lstdefinestyle{mystyle}{
    backgroundcolor=\color{backcolour},   
    commentstyle=\color{codegreen},
    keywordstyle=\color{magenta},
    numberstyle=\tiny\color{codegray},
    stringstyle=\color{codepurple},
    basicstyle=\ttfamily\footnotesize,
    breakatwhitespace=false,         
    breaklines=true,                 
    captionpos=b,                    
    keepspaces=true,                 
    numbers=left,                    
    numbersep=5pt,                  
    showspaces=false,                
    showstringspaces=false,
    showtabs=false,                  
    tabsize=2
}

\lstset{style=mystyle}

\title{集成电路实验一\\数字系统综合实验-Quartus II、Modelsim、FPGA操作入门}
\date{\today}
\author{第一组 物理2001孙陶庵 202011010101 2023/12/05物电院A116实验}
\begin{document}
\begin{spacing}{2.0}
\tableofcontents
\maketitle


\section{实验原理}
\section{实验内容}
\subsection{实验步骤}
使用Quartus II编程,写出一份可以操控电路板上led灯的程序,并调控参数观察现象。\\
实验步骤:\\
1.	电脑创建程序写出程序\\
2.	编译,debug\\
3.	使用Modelsim 仿真并观察波形, 确认无误后将电路板接进电脑\\
4.	使用Quartus II 将程序导入电路板\\

\subsection{实验结果}


\subsection{拓展实验}
读懂代码,试修改代码,完成以下实验步骤。\\
a.	如何改变流水灯的移动速度?\\
b.	如何改变流水灯的移动方向?\\
c.	如何移动两个LED灯?\\
d.	分别用$Key\_1$、 $Key\_2$、 $Key\_3$三个按键来控制移动频率1Hz、2Hz和移动方向\\
e.	查看资源占有率及RTL视图\\
拓展实验结果:\\
a.LED的移动速度是通过计时器 counter 和计时结束标识 $end\_cnt$ 来控制的。所以可以透过修改这一部分来改变移动速度
\begin{lstlisting}
    always @(posedge clk or negedge rst_n or negedge key_1 or negedge key_2)
    if (!rst_n)
        counter <= 0;
        end_cnt <= END_CNT_1S;  // 初始设置为1秒的计时结束标识
    else if (!key_1)  // 按下后周期变为1S
        end_cnt <= END_CNT_1S;
    else if (!key_2)  // 按下后周期变为2S
        end_cnt <= END_CNT_2S;
    else if (counter == end_cnt)
        counter <= 0;
    else 
        counter <= counter + 1'b1;

\end{lstlisting}

通过按下 $Key\_1$,LED的速度将被设置为1秒一次,而按下$Key\_2$,LED的速度将被设置为2秒一次。这是因为 $end\_cnt$ 在这里用于确定计时器何时达到结束状态,进而改变LED的状态。

因此,如果希望改变LED的移动速度,可以通过按下不同的按键来选择不同的计时结束标识。\\

b.LED的移动方向是通过按键输入 $key\_3$ 控制的。按下 $key\_3$ 时,direction寄存器会取反,从而改变LED的移动方向。

\begin{lstlisting}

always @(posedge clk or negedge rst_n or negedge key_1 or negedge key_2 or negedge key_3)
    if (!rst_n)
    begin
        counter <= 0;
        end_cnt <= END_CNT_1S;  // 初始设置为1秒的计时结束标识
        direction <= 1'b0;  // 初始方向为0
    end
    else if (!key_1)  // 按下后周期变为1S
        end_cnt <= END_CNT_1S;
    else if (!key_2)  // 按下后周期变为2S
        end_cnt <= END_CNT_2S;
    else if (!key_3)  // 按下后改变方向
        direction <= ~direction;  // 取反方向
    else if (counter == end_cnt)
        counter <= 0;
    else 
        counter <= counter + 1'b1;

\end{lstlisting}
当按下 $key\_3$ 时,direction 的值将翻转,从而改变LED的移动方向。这是通过 direction <= ~direction; 这一语句实现的。

因此,如果希望改变LED的移动方向,只需按下 $key\_3$ 即可。

c.
可以尝试使用两个计时器和两个计时结束标识来分别控制两个LED的移动。
\begin{lstlisting}

module dual_led (
    input wire clk,
    input wire rst_n,
    input wire key_1,
    input wire key_2,
    input wire key_3,
    output reg[3:0] led1,
    output reg[3:0] led2
);

reg [25:0] counter1;    // 1秒计数器1
reg [25:0] counter2;    // 1秒计数器2
reg [25:0] end_cnt1;    // 结束时间周期1
reg [25:0] end_cnt2;    // 结束时间周期2
reg direction1;         // 方向1,0左1右
reg direction2;         // 方向2,0左1右

parameter END_CNT_1S = 4999999;  // 1秒计数结束标识
parameter END_CNT_2S = 9999999;  // 2秒计数结束标识

always @(posedge clk or negedge rst_n or negedge key_1 or negedge key_2 or negedge key_3)
    if (!rst_n)
    begin
        counter1 <= 0;
        end_cnt1 <= END_CNT_1S;
        direction1 <= 1'b0;  // 初始方向为0
    end
    else if (!key_1)
        end_cnt1 <= END_CNT_1S;
    else if (!key_2)
        end_cnt1 <= END_CNT_2S;
    else if (!key_3)
        direction1 <= ~direction1;

    // Counter1 logic
    if (counter1 == end_cnt1)
        counter1 <= 0;
    else
        counter1 <= counter1 + 1'b1;

always @(posedge clk or negedge rst_n or negedge key_1 or negedge key_2 or negedge key_3)
    if (!rst_n)
    begin
        counter2 <= 0;
        end_cnt2 <= END_CNT_1S;
        direction2 <= 1'b0;  // 初始方向为0
    end
    else if (!key_1)
        end_cnt2 <= END_CNT_1S;
    else if (!key_2)
        end_cnt2 <= END_CNT_2S;
    else if (!key_3)
        direction2 <= ~direction2;

    // Counter2 logic
    if (counter2 == end_cnt2)
        counter2 <= 0;
    else
        counter2 <= counter2 + 1'b1;

always @(posedge clk or negedge rst_n)
begin
    if (!rst_n)
    begin
        led1 <= 4'b0001;
        led2 <= 4'b0001;
    end
    else
    begin
        // LED1 logic
        case({counter1 == end_cnt1, direction1})
            2'b10: led1 <= led1 << 1;
            2'b01: led1 <= led1 >> 1;
            default: led1 <= led1;
        endcase

        // LED2 logic
        case({counter2 == end_cnt2, direction2})
            2'b10: led2 <= led2 << 1;
            2'b01: led2 <= led2 >> 1;
            default: led2 <= led2;
        endcase
    end
end

endmodule
\end{lstlisting}
分别由两个计时器 counter1 和 counter2 控制。按键 $key\_3$ 仍然用于改变方向,
而按键 $key\_1$ 和 $key\_2$ 分别用于选择1秒和2秒的计时结束标识。LED的移动逻辑分别由两个独立的 case 语句控制。\\
d.
$Key\_1$ 选择1Hz的计时结束标识和向左移动。
$Key\_2$ 选择2Hz的计时结束标识和向左移动。
$Key\_3$ 按下则切换移动方向。























\end{spacing}{}
%\bibliographystyle{IEEEtran}
%\bibliography{re1}

\end{document}